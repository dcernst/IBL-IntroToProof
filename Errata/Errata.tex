\documentclass[11pt]{article}%{amsart}

\usepackage[margin=1in]{geometry}
\usepackage{url}
\usepackage{amsmath}
\usepackage{amsthm}
\usepackage{amssymb}
\usepackage[breaklinks]{hyperref}
\usepackage{color}
\hypersetup{
	colorlinks=true,
	linkcolor=darkblue,
	anchorcolor=darkblue,
	citecolor=darkblue,
	pagecolor=darkblue,
	urlcolor=darkblue,
	pdftitle={},
	pdfauthor={},
    bookmarksnumbered
}
\definecolor{darkblue}{rgb}{0, 0, .6}

\setlength{\parindent}{0pt}
\setlength{\fboxsep}{10pt}

\newcommand{\blankline}{\pagebreak[2]\vspace{.5\baselineskip}}

\DeclareMathOperator{\rel}{rel}
\newcommand{\Rel}{\operatorname{Rel}}

%%%%%%%%%%%%%%%%%%%

\begin{document}

\title{An Introduction to Proof via Inquiry-Based Learning\\
Errata}
\author{Dana C.~Ernst}
\date{\today}

\maketitle

All errors/improvements listed below have been corrected in the version of the book that has been compiled from the current source.  Page numbers below reference the print version of the textbook published by AMS/MAA Press.

\blankline

If you think you have found any errors not already listed here, please submit an issue on \href{https://github.com/dcernst/IBL-IntroToProof/issues}{GitHub} or send me an email at \url{dana.ernst@nau.edu}.

\begin{itemize}
\item Page x, line $-5$: Replaced ``undoubtably" with ``undoubtedly". Although, perhaps the former is acceptable. [Pietro Monticone]
\item Page 11: Added some commentary to clarify use of ``theorem'' vs ``proposition''. [Inspired by David Austin]
\item Page 14, line 7: Deleted extra ``the".
\item Page 15, line $-7$: Added ``are" between ``we" and ``defining". [Elizabeth Shute]
\item Page 19, lines 5--6: Removed the potentially-misleading sentence ``Moreover, the converse and inverse of a conditional proposition do not necessarily have the same truth value as each other.'' [Inspired by David Austin]
\item Page 31, Problem~2.79: Removed spurious ``a" in ``\ldots is a not a nugget\ldots". [Eliza Todd]
\item Page 52, line 1: Replaced ``isn" with ``is".
\item Page 54, Theorem~4.2: It would be nice if Item~(ii) began ``for all $k\geq 1$".  This is an instance of an implicit universal quantifier for a conditional statement.  In this case, the implicit universe of discourse is the set of natural numbers.  However, including ``for all $k\geq 1$," leads to a nice parallelism with Theorem~4.9. [Ben Ford]
\item Page 56, Theorem~4.9: Item~(ii) should start with ``for all $k\geq a$,". [Ben Ford]
\item Page 57, Skeleton Proof~4.10: In Item~(ii), it should say ``For all $k\geq a$, if $P(k)$ is true, then $P(k+1)$ is true." as opposed to ``For all $k\in\mathbb{Z}\ldots$" [Ben Ford]
\item Page 59, line 17: Removed spurious comma between ``need" and ``is".
\item Page 59, Skeleton Proof~4.26: The stated goal in the inductive step contained an extra universal quantifier.  The extra phrase ``for each $k \in \mathbb{N}$" has been removed.  In addition, the last sentence of the skeleton proof contained a copy-paste error from the previous skeleton proof.  The phrase ``for all integers $n \ge a$" has been replaced with ``for all $n\in\mathbb{N}$". [Blake Farman]
\item Page 60, Problem~4.32: Replaced ``chessboard" with ``grid".
\item Page 60, Problem~4.33: Deleted extra ``the". [Ellie Nichols]
\item Page 60, Problem~4.33: In Item~(i), the final string is missing a leading one.  It should be ``$011101 \to 111101$". [Ben Ford]
\item Page 63, line 1: Missing the word ``the" between ``into" and ``structure".
\item Page 63, line $-6$: Replaced the word ``subtraction" with ``multiplication". 
\item Page 64: Retooled the paragraph above Theorem~5.2 in an attempt to clarify the existence versus uniqueness of additive and multiplicative identities. [inspired by suggestion from Alistair Windsor]
\item Page 64, Theorem~5.2: Retooled theorem statement in an attempt to clarify the existence versus uniqueness of additive identity. [Inspired by suggestion from Alistair Windsor]
\item Page 64, Theorem~5.3: Retooled theorem statement in an attempt to clarify the existence versus uniqueness of multiplicative identity. [Inspired by suggestion from Alistair Windsor]
\item Page 64: Retooled paragraph prior to Theorem~5.4 in an attempt to clarify the existence versus uniqueness of additive and multiplicative inverses. In addition, a new paragraph was added after Theorem~5.5 to help further clarify the issue. Also added a comment about $-0=0$. [Inspired by suggestion from Alistair Windsor]
\item Page 65, line $-9$: Corrected the definition of rational numbers, which had omitted the requirement that $a$ and $b$ be integers. [David Austin]
\item Page 67, Theorem~5.21: Added ``Moreover, if $a,b\in \mathbb{R}\setminus\{0\}$ with $a<b$, then $b^{-1}<a^{-1}$." This should probably be a separate result, but I didn't want to mess up the numbering.
\item Page 82, line $-1$: ``counterexample" should be replaced with ``contradiction". 
\item Page 101, Problem~7.47: Added the word ``nonempty" to clarify that we are not considering the empty graph.
\item Page 103, Theorem~7.59: Removed the word ``nonempty" as the theorem holds even if $A$ is the empty set. [David Deville]
\item Page 103, Problem~7.60: In light of previous item, this problem has been replaced  with ``In the previous theorem, what is $A/\mathord\sim$ if $A$ is the empty set?"
\item Page 103, Problem~7.61: Replace ``Find the partition determined by $\Rel(\sim)$" with ``Find $A/\mathord\sim$ and verify that it is a partition".
\item Page 104, Problem~7.65: Replaced ``$\sim$" with ``$R$".
\item Page 104, Theorem~7.68: Removed the word ``nonempty" as the theorem holds even if $A$ is the empty set. [David Deville]
\item Page 104, Problem~7.69: In light of previous item, this problem has been replaced  with ``In the previous theorem, what is $R_{\Omega}$ if $A$ is the empty set?"
\item Page 104, Problem~7.72: In light of changes to Theorem~7.59, Problem~7.60, Theorem~7.68, and Problem~7.69, this problem has been replaced  with ``Let $A=\{a,b,c\}$. If possible, find an example of collection $\Omega$ of nonempty subsets of $A$ such that $R_{\Omega}$ is an equivalence relation on $A$ but $\Rel(R_{\Omega})\neq \Omega$.  If such an example is impossible, explain why."
\item Page 105, Corollary~7.74: As stated the corollary is incorrect. However, including the hypothesis ``$a\in\rel(a)$ for all $a\in A$" makes the statement true. In addition, the word ``nonempty" has been removed as teh result still holds even if $A$ is the empty set. I think it is incorrect to refer to this result as a ``Corollary", so it has been renamed ``Theorem". As a result, the reference to the result in Example~7.80 and the paragraph prior to Theorem~7.81 have been updated. A short paragraph about the new Theorem~7.74 has been added. [Blake Farman, David Deville]
\item Page 106, Example~7.80 and paragraph prior to Theorem~7.81: ``Corollary~7.74" replaced with ``Theorem~7.74."
\item Page 113, Figure~8.1: Removed extra comma in caption.
\item Page 120, Problem~8.35. There is no straightforward way to prove that the function in Part~(d) is surjective, so I added the following comment: ``\emph{Note:} You are probably not in a position to write a careful argument for surjectivity for Part~(d)."  
\item Page 128, Problem~8.66: It is impossible to find an example for Part~(a) with the given sets.  ``Let $X=\{a,b\}$ and $Y=\{1,2\}$'' has been replaced with ``Complete each of the following. Consider using finite sets and drawing a function diagram to define your functions.'' [Blake Farman]
\item Page 131, line 7: Corrected the spelling of ``Definition." [Pietro Monticone]
\item Page 154, line 1: Added the word ``us" between ``tells" and ``that".
\item Page 158, line 1: Added the word ``look" between ``you" and ``at". [Roy St.~Laurent]
\item Page 161, line 11: Removed spurious ``$)$". [Pietro Monticone]
\item Page 163: Added alternative definition of ``proposition'' that aligns with Definition~2.16. [Inspired by David Austin]
\end{itemize}

\end{document}