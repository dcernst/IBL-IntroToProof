\documentclass[11pt]{article}

\usepackage{amsfonts}
\usepackage{amsmath}
\usepackage{wasysym}
\usepackage{amssymb}
\usepackage{stmaryrd}
\usepackage{amsthm}
\usepackage{fancyhdr}
\usepackage[margin=1in]{geometry}
\usepackage[hang,flushmargin,symbol*]{footmisc}
\usepackage{color}
\definecolor{darkblue}{rgb}{0, 0, .6}
\definecolor{grey}{rgb}{.7, .7, .7}
\usepackage[breaklinks]{hyperref}
\hypersetup{
	colorlinks=true,
	linkcolor=darkblue,
	anchorcolor=darkblue,
	citecolor=darkblue,
	pagecolor=darkblue,
	urlcolor=darkblue,
	pdftitle={},
	pdfauthor={}
}

\newcommand{\dom}{\operatorname{Dom}}
\newcommand{\codom}{\operatorname{Codom}}
\newcommand{\range}{\operatorname{Rng}}

\pagestyle{fancy}

\lhead{\scriptsize Notes for an Introduction to Proof Course}
\rhead{\scriptsize Instructor: {Paul Ellis}}
%\lfoot{\scriptsize This work is an adaptation of notes written by Stan Yoshinobu of Cal Poly and Matthew Jones of California State University, Dominguez Hills.}
\cfoot{}

\renewcommand{\headrulewidth}{0.4pt}
\renewcommand{\footrulewidth}{0.4pt}

\theoremstyle{definition}
\newtheorem{theorem}{Theorem}[section]
\newtheorem{acknowledgement}[theorem]{Acknowledgement}
\newtheorem{algorithm}[theorem]{Algorithm}
\newtheorem{axiom}[theorem]{Axiom}
\newtheorem{case}[theorem]{Case}
\newtheorem{claim}[theorem]{Claim}
\newtheorem{conclusion}[theorem]{Conclusion}
\newtheorem{condition}[theorem]{Condition}
\newtheorem{conjecture}[theorem]{Conjecture}
\newtheorem{corollary}[theorem]{Corollary}
\newtheorem{criterion}[theorem]{Criterion}
\newtheorem{definition}[theorem]{Definition}
\newtheorem{example}[theorem]{Example}
\newtheorem{exercise}[theorem]{Exercise}
\newtheorem{journal}[theorem]{Journal}
\newtheorem{lemma}[theorem]{Lemma}
\newtheorem{notation}[theorem]{Notation}
\newtheorem{problem}[theorem]{Problem}
\newtheorem{proposition}[theorem]{Proposition}
\newtheorem{remark}[theorem]{Remark}
\newtheorem{solution}[theorem]{Solution}
\newtheorem{summary}[theorem]{Summary}
\newtheorem{question}[theorem]{Question}

\begin{document}

\addtocounter{section}{5}

\begin{section}{Relations and Functions}

\addtocounter{subsection}{5}
\addtocounter{theorem}{109}

\begin{subsection}{Image and inverse Image}

\begin{definition}
If $f:X\to Y$ is a function and $A\subseteq X$, then the \textbf{image of $A$ under $f$} is
$$f[A]=\{y\in Y: (\exists x\in A)(f(x)=y)\}$$
\end{definition}

\begin{definition}

If $f:X\to Y$ is a function and $C\subseteq Y$, then the
\textbf{inverse image (or preimage) of $C$ under $f$} is
$$f^{-1}[C]=\{x\in X: f(x)\in C\}$$

\end{definition}

\begin{exercise}

Let $f:\mathbb{R}\to\mathbb{R}$ be defined by $f(x)=3x^2-4$.
Find all of the following. You probably want to draw the graph
of $f(x)$ first.
\begin{enumerate}
\item $f[[-2,4]]$
\item $f[(-2,4)]$
\item $f^{-1}[[-10,1]]$
\item $f^{-1}[(-3,3)]$
\item $f[\emptyset]$
\item $f[\mathbb{R}]$
\item $f^{-1}[\emptyset]$
\item $f^{-1}[\mathbb{R}]$
\item Find two non-empty subsets $A$, $B$ of $\mathbb{R}$ such that $A\cap B=\emptyset$ and $f^{-1}[A]=f^{-1}[B]$
\item Find two non-empty subsets $A$, $B$ of $\mathbb{R}$ such that $A\cap B=\emptyset$ and $f[A]=f[B]$
\end{enumerate}

\end{exercise}

\begin{theorem}
Let $f:X\to Y$ be a function and let $A\subseteq X$, $B\subseteq X$.
\begin{enumerate}
\item[(a)] If $A\subseteq B$, then $f[A]\subseteq f[B]$
\item[(b)] $f[A\cup B]=f[A]\cup f[B]$
\item[(c)] $f[A\cap B]\subseteq f[A]\cap f[B]$ (When are they equal?)
\end{enumerate}
\end{theorem}

\begin{theorem}
Let $f:X\to Y$ be a function and let $C\subseteq Y$, $D\subseteq Y$.
\begin{enumerate}
\item[(a)] If $C\subseteq D$, then $f^{-1}[C]\subseteq f^{-1}[D]$
\item[(b)] $f^{-1}[C\cup D]=f^{-1}[C]\cup f^{-1}[D]$
\item[(c)] $f^{-1}[C\cap D]=f^{-1}[C]\cap f^{-1}[D]$
\end{enumerate}
\end{theorem}

\begin{theorem}
Let $f:X\to Y$ be a function and let $A\subseteq X$, $C\subseteq Y$.
\begin{enumerate}
\item[(a)] $A\subseteq f^{-1}[f[A]]$ (When are they equal?)
\item[(b)] $f[f^{-1}[C]]\subseteq C$ (When are they equal?)
\end{enumerate}
\end{theorem}

\end{subsection}

\end{section}

\end{document}
