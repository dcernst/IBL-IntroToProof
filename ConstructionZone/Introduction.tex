\chapter{Introduction}\label{chap:intro}

\begin{section}{What is This Book All About?}%new title?

The foundations of mathematics refers to logic and set theory; the axioms of number and space.  Also, it refers to an introduction to the techniques of proof, and at a larger level the process of \emph{doing mathematics}.  Proof is central to doing mathematics.

Up to this point, it is likely that your experience of mathematics has been about using formulas and algorithms. That is only one part of mathematics. Mathematicians do much more than just use formulas.  Mathematicians experiment, make conjectures, write definitions, and prove theorems.  While engaging with the material contained in this book, we will learn about doing all of these things. The focus of this book is on learning to \emph{do} mathematics. 

%What will learning from this book require?  Daily practice.  
Just like learning to play an instrument or sport, you will have to learn new skills and ideas.  Sometimes you will feel good, sometimes frustrated.  You will probably go through a range of feelings from being exhilarated, to being stuck.  Figuring it out, victories, defeats, and all that is part of real life is what you can expect.  Most importantly it will be rewarding.  Learning mathematics requires dedication.  It will require that you be patient despite periods of confusion.  It will require that you persevere in order to understand.  As the author of this text, I am here to guide you, but I cannot do the learning for you, just as a music teacher cannot move your fingers and your heart for you.  Only you can do that.  
%I can give suggestions, structure the course to assist you, and try to help you figure out how to think through things. 
Do your best, be prepared to put in a lot of time, and do all the work. 
%Ask questions in class, ask questions in office hours, and ask your classmates questions.  
When you work hard and you come to understand, you feel good about yourself.  In the meantime, you have to believe that your work will pay off in intellectual development.

%The focus of this book is on learning to \emph{do} mathematics.  Therefore, class time will be devoted to working on problems, and especially on students presenting conjectures and proofs to the class, asking questions of presenters in order to understand their work and their thinking, and sharing and clarifying our thinking and understanding of each other's ideas.  

Your progress will be fueled by your ability to wrestle with mathematical ideas and to prove theorems.  As you work through the book, you will find that you have ideas for proofs, but you are unsure of them.  Do not be afraid to tinker and make mistakes.  You can always revisit your work as you become more proficient. Do not expect to do most things perfectly on your first attempt. 

%The class is fueled by your ability to prove theorems and share your ideas.  As we progress, you will find that you have ideas for proofs, but you are unsure of them.  In that case, you can either bring your idea to the class, or you can bring it to office hours.  By coming to office hours, you have a chance to refine your ideas and get individual feedback before bringing them to the class.  The more you use office hours, the more you will learn.  If the whole class is stuck, we can work on some ego-booster problems to get your ideas flowing.

This is a very exciting time in your mathematical career.  It is where you learn what mathematics is really about!

\epigraph{The mathematician does not study pure mathematics because it is useful; he studies it because he delights in it, and he delights in it because it is beautiful.}{\emph{Henri Poincar\'e}}

\end{section}

\begin{section}{An Inquiry-Based Approach}

This is not a typical textbook or one in which mimicking prefabricated examples will lead you to success. You will be expected to work actively to construct your own understanding of the topics at hand. 
%You will be expected to work actively to construct your own understanding of the topics at hand with the readily available help of me and your classmates. 
Many of the concepts you learn and problems you work on will be new to you and ask you to stretch your thinking. You will experience \emph{frustration} and \emph{failure} before you experience \emph{understanding}. This is part of the normal learning process. If you are doing things well, you should be confused on a regular basis. The material is too rich for a human being to completely understand it immediately. Your viability as a professional in the modern workforce depends on your ability to embrace this learning process and make it work for you.

\epigraph{Don't fear failure.  Not failure, but low aim, is the crime. In great attempts it is glorious even to fail.}{\emph{Bruce Lee}}

In order to promote a more active participation in your learning, this book adheres to an educational philosophy called inquiry-based learning (IBL).  Loosely speaking, IBL is a student-centered method of teaching mathematics that engages students in sense-making activities.  Students are given tasks requiring them to solve problems, conjecture, experiment, explore, create, and communicate.  Rather than showing facts or a clear, smooth path to a solution, this book will guide and mentor you via well-crafted problems through an adventure in mathematical discovery.  According to \href{https://www.colorado.edu/eer/sites/default/files/attached-files/laursenrasmussencommentaryauthorversion0219.pdf}{Laursen and Rasmussen (2019)}, the Four Pillars of IBL are:
\begin{itemize}
\item Students engage deeply with coherent and meaningful mathematical tasks.
\item Students collaboratively process mathematical ideas.
\item Instructors inquire into student thinking.
\item Instructors foster equity in their design and facilitation choices.
\end{itemize}

While progressing through this book, you will be asked to engage with the material and to produce mathematics. The best way to learn mathematics is by doing mathematics.  Someone cannot master a musical instrument or a martial art by simply watching, and in a similar fashion, you cannot master mathematics by simply watching; you must do mathematics!

%Much of the course will be devoted to students presenting their proposed solutions or proofs on the board and a significant portion of your grade will be determined by how much mathematics you produce.  I use the word \emph{produce} because I believe that the best way to learn mathematics is by doing mathematics.  Someone cannot master a musical instrument or a martial art by simply watching, and in a similar fashion, you cannot master mathematics by simply watching; you must do mathematics!

In any act of creation, there must be room for experimentation, and thus allowance for mistakes, even failure. Mistakes are inevitable, and they should not be an obstacle to further progress. Productive struggle and mistakes provide opportunities for growth.  It is normal to struggle and be confused as you work through new material. Accepting that means you can keep working even while feeling stuck, until you overcome and reach even greater accomplishments.

%In any act of creation, there must be room for experimentation, and thus allowance for mistakes, even failure. A key goal of our community is that we support each other---sharpening each other's thinking but also bolstering each other's confidence---so that we can make failure a productive experience. Mistakes are inevitable, and they should not be an obstacle to further progress. It's normal to struggle and be confused as you work through new material. Accepting that means you can keep working even while feeling stuck, until you overcome and reach even greater accomplishments.

\epigraph{You will become clever through your mistakes.}{\emph{German Proverb}}

Furthermore, it is important to understand that solving genuine problems is difficult and takes time.  You should not expect to complete each problem in 10 minutes or less.  Sometimes you might have to stare at the problem for an hour before even understanding how to get started.

%In this course, everyone will be required to
%\begin{itemize}
%\item read and interact with course notes and textbook on your own;
%\item write up quality solutions/proofs to assigned problems;
%\item present solutions/proofs on the board to the rest of the class;
%\item participate in discussions centered around a student's presented solution/proof;
%\item call upon your own prodigious mental faculties to respond in flexible, thoughtful, and creative ways to problems that may seem unfamiliar on first glance.
%\end{itemize}
%As the semester progresses, it should become clear to you what the expectations are.

\epigraph{Tell me and I forget, teach me and I may remember, involve me and I learn.}{\emph{Benjamin Franklin}}

\end{section}

\begin{section}{Rights of the Learner}\label{sec:Rights of the Learner}
As a reader of this textbook, you have the right:
\begin{enumerate}
\item to be confused,
\item to make a mistake and to revise your thinking,
\item to speak, listen, and be heard, and
\item to enjoy doing mathematics.
\end{enumerate}

\epigraph{You may encounter many defeats, but you must not be defeated.}{\emph{Maya Angelou}}
	
\end{section}

\begin{section}{Your Toolbox, Questions, and Observations}

We will develop a list of \emph{tools} that will help you understand and do mathematics. Your job is to keep a list of these tools, and it is suggested that you keep a running list someplace.

It is of utmost importance that you work to understand every proof.  Questions---asked to your instructor, your peers, and yourself---are often your best tool for determining whether you understand a proof.  
%Here are some sample questions that apply to any proof that you should be prepared to ask of yourself:
%\begin{itemize}
%\item What method(s) of proof are you using?
%\item What form will the conclusion take?
%\item How did you know to set up that [equation, set, whatever]?
%\item How did you figure out what the problem was asking?
%\item Was this the first thing you tried?
%\item Can you explain how you went from this line to the next one?
%\item What were you thinking when you introduced this?
%\item Could we have \ldots instead?
%\item Would it be possible to \ldots?
%\item What if \ldots?
%\end{itemize}
Another way to help you process and understand a proof is to try and make observations and connections between different ideas, proof statements and methods, and to compare various approaches. Observations might sound like some of the following:
\begin{itemize}
\item When I tried this proof, I thought I needed to \ldots But I didn't need that, because \ldots
\item I think that \ldots is important to this proof, because \ldots
\item When I read the statement of this theorem, it seemed similar to this earlier theorem. Now I see that it [is/isn't] because \ldots
\end{itemize}

\end{section}

\begin{section}{Structure of the Textbook}

As you read the textbook, you will be required to digest the material in a meaningful way.  It is your responsibility to read and understand new definitions and their related concepts.  
%However, you will be supported in this sometimes difficult endeavor. 
In addition, you will be asked to complete problems aimed at solidifying your understanding of the material.  Most importantly, you will be asked to make conjectures, produce counterexamples, and prove theorems. All of these tasks will almost always be challenging.

The items labeled as \textbf{Definition} and \textbf{Example} are meant to be read and digested.  However, the items labeled as \textbf{Problem}, \textbf{Theorem}, \textbf{Corollary}, and \textbf{Lemma} require action on your part.  Items labeled as \textbf{Problem} are sort of a mixed bag. Some Problems are computational in nature and aimed at improving your understanding of a particular concept while others ask you to provide a counterexample for a statement if it is false or to provide a proof if the statement is true. Items with the \textbf{Theorem}, \textbf{Corollary}, and \textbf{Lemma} designation are mathematical facts and the intention is for you to produce a valid proof of the given statement.  A lemma is a minor result whose sole purpose is to help in proving a theorem. The main difference between a theorem and a corollary is that corollaries are typically statements that follow quickly from a previous theorem.  In general, you should expect corollaries to have very short proofs.  However, that does not mean that you cannot produce a more lengthy yet valid proof of a corollary.

It is important to point out that there are very few examples in this book.  This is intentional.  One of the goals of the items labeled as \textbf{Problem} is for you to produce the examples. Lastly, there are many situations where you will want to refer to an earlier definition, problem, theorem, corollary, or lemma.  In this case, you should reference the statement by number, but it is also helpful to the reader to summarize the statement you are citing.  For example, you might write something like, ``By Theorem~\ref{thm:two consecutive ints}, the sum of two consecutive integers is odd, and so\ldots."

\end{section}

\begin{section}{Some Minimal Guidance}
Especially in the opening sections, it will not be clear what facts from your prior experience in mathematics we are ``allowed" to use.  Unfortunately, addressing this issue is difficult and is something we will sort out along the way.  However, in general, here are some minimal guidelines to keep in mind.  

First, there are times when we will need to do some basic algebraic manipulations.  You should feel free to do this whenever the need arises.  But you should show sufficient work along the way.  You do not need to write down justifications for basic algebraic manipulations (e.g., adding 1 to both sides of an equation, adding and subtracting the same amount on the same side of an equation, adding like terms, factoring, basic simplification, etc.).  

On the other hand, you do need to make explicit justification of the logical steps in a proof.  As stated in the previous section, when necessary, you should cite a previous definition, theorem, etc.

Unlike the experience many of you had writing proofs in your high school geometry class, our proofs should be written in complete sentences.  You should break sections of a proof into paragraphs and use proper grammar.  There are some pedantic conventions for doing this that will be pointed out along the way.  Initially, this will be an issue that you may struggle with, but you will get the hang of it.

Ideally, you should rewrite the statement of a theorem before you start the proof.  Moreover, for the sake of the reader of your proof, you should indicate where the proof begins by writing ``\emph{Proof.}" at the beginning.  Also, you should conclude your proofs with the standard ``proof box" (i.e., $\square$ or $\blacksquare$), which is typically right-justified.

Lastly, every time you write a proof, you need to make sure that you are making your assumptions crystal clear.  Sometimes there will be some implicit assumptions that we can omit, but at least in the beginning, you should get in the habit of stating your assumptions up front.  Typically, these statements will start off ``Assume\ldots", ``Suppose\ldots", or ``Let\ldots".  

This should get you started.  We will discuss more as the semester progresses.  Now, go have fun and start exploring mathematics!

\epigraph{If you want to sharpen a sword, you have to remove a little metal.}{\emph{Unknown}}

\end{section}