\documentclass[12pt,oneside]{book}

\usepackage[scale=2]{ccicons}
\usepackage{caption}
\usepackage[labelfont={}]{subcaption}
\captionsetup{subrefformat=parens}
\usepackage{booktabs}
\usepackage{epigraph}
\usepackage{enumitem}
\usepackage{newclude}
\usepackage{multicol}
\usepackage{tabu}
\usepackage[table]{xcolor}
\usepackage{tikz}
\usetikzlibrary{arrows,automata,positioning,fit,shapes}
\usetikzlibrary{decorations.markings}
\usetikzlibrary{decorations.pathreplacing}
\usetikzlibrary{matrix}
\usepackage[framemethod=TikZ]{mdframed}
\usepackage{rotating}
\usepackage[notextcomp]{kpfonts} 
\usepackage{graphicx}
\usepackage{eurosym}
\usepackage{amsfonts}
\usepackage{mathtools}
\usepackage{amsmath}
\usepackage{amssymb}
\usepackage{stmaryrd}
\usepackage{wasysym}
\usepackage{amsthm}
\usepackage[margin=1in]{geometry}
\usepackage[hang,flushmargin]{footmisc}
\usepackage{color}
\usepackage{longfbox}
\usepackage[most]{tcolorbox}
\usepackage[breaklinks]{hyperref}
\hypersetup{
	colorlinks=true,
	linkcolor=darkblue,
	anchorcolor=darkblue,
	citecolor=darkblue,
	pagecolor=darkblue,
	urlcolor=darkblue,
	pdftitle={},
	pdfauthor={},
    bookmarksnumbered
}

%%% Colors %%%

\definecolor{darkblue}{rgb}{0, 0, .6}
\definecolor{grey}{rgb}{.6, .6, .6}
\definecolor{lightgrey}{rgb}{.85, .85, .85}
\definecolor{naugreen}{cmyk}{.43,0,.34,.38}
\definecolor{naublue}{cmyk}{1,.72,0,.32}
\definecolor{mediterranean}{cmyk}{.67,0,.08,.3}
\definecolor{rose}{cmyk}{0,1.00,.20,0}
\definecolor{darkorchid}{cmyk}{.6,.9,0,.05}
\definecolor{butterfly}{cmyk}{.95,.59,0,.10}
\definecolor{springgreen}{cmyk}{1.00,0,.70,.02}
\definecolor{darkred}{cmyk}{0,1,1,.5}
\definecolor{nectarine}{cmyk}{0,0.70,1.00,0}
\definecolor{icyblue}{cmyk}{.84,.25,0,.06}
\definecolor{orange}{RGB}{255,102,0}
\definecolor{ggreen}{RGB}{0,153,0}
\definecolor{darkblue}{RGB}{0,0,255}
\definecolor{purple}{RGB}{153,51,255}
\definecolor{turq}{RGB}{72,209,204}
\definecolor{lightskyblue}{cmyk}{.4,.11,0,.2}
\definecolor{softplum}{cmyk}{.02,.45,0,.68}%.37
\definecolor{lightorange}{cmyk}{0,.45,.59,.02}
\definecolor{brickorange}{cmyk}{0,.59,.59,.20}
\definecolor{deeppink}{cmyk}{0,.6,.3,.3}
\definecolor{orchid}{cmyk}{0,.49,.02,.05}

\definecolor{grey1}{RGB}{247,247,247}
\definecolor{grey2}{RGB}{204,204,204}
\definecolor{grey3}{RGB}{150,150,150}
\definecolor{grey4}{RGB}{99,99,99}
\definecolor{grey5}{RGB}{37,37,37}

\usepackage{fancyhdr}
\pagestyle{fancy}
\lhead{\leftmark}
\chead{}
\rhead{}
\lfoot{}
\cfoot{\thepage}
\rfoot{}

\theoremstyle{definition}
\newtheorem{theorem}{Theorem}[chapter]
\newtheorem{acknowledgement}[theorem]{Acknowledgement}
\newtheorem{algorithm}[theorem]{Algorithm}
\newtheorem{axiom}[theorem]{Axiom}
\newtheorem{case}[theorem]{Case}
\newtheorem{claim}[theorem]{Claim}
\newtheorem{conclusion}[theorem]{Conclusion}
\newtheorem{condition}[theorem]{Condition}
\newtheorem{conjecture}[theorem]{Conjecture}
\newtheorem{corollary}[theorem]{Corollary}
\newtheorem{criterion}[theorem]{Criterion}
\newtheorem{definition}[theorem]{Definition}
\newtheorem{example}[theorem]{Example}
\newtheorem{exercise}[theorem]{Exercise}
\newtheorem{journal}[theorem]{Journal}
\newtheorem{lemma}[theorem]{Lemma}
\newtheorem{notation}[theorem]{Notation}
\newtheorem{problem}[theorem]{Problem}
\newtheorem{proposition}[theorem]{Proposition}
\newtheorem{remark}[theorem]{Remark}
\newtheorem{solution}[theorem]{Solution}
\newtheorem{summary}[theorem]{Summary}
\newtheorem{skeleton}[theorem]{Skeleton Proof}
\newtheorem{activity}[theorem]{Activity}
\newtheorem{intuitivedef}[theorem]{Intuitive Definition}
\newtheorem{question}[theorem]{Question}
\newtheorem{fieldaxioms}[theorem]{Field Axioms}
\newtheorem{axioms}[theorem]{Axioms}
\newtheorem{orderaxioms}[theorem]{Order Axioms}
\newtheorem{completenessaxiom}[theorem]{Completeness Axiom}
\newtheorem{additionalaxioms}[theorem]{Additional Order Axioms}

\newcommand{\dom}{\operatorname{Dom}}
\newcommand{\codom}{\operatorname{Codom}}
\newcommand{\range}{\operatorname{Rng}}
\newcommand{\lcm}{\operatorname{lcm}}
\newcommand{\card}{\operatorname{card}}
\newcommand{\rel}{\operatorname{rel}}
\newcommand{\Rel}{\operatorname{Rel}}
\newcommand{\logeq}{\Longleftrightarrow}
\renewcommand{\implies}{\Longrightarrow}

\setlength{\epigraphwidth}{0.5\textwidth}
\renewcommand{\sourceflush}{flushleft}

\mdfdefinestyle{skeleton}{roundcorner=5pt,leftmargin=5pt,rightmargin=5pt,backgroundcolor = black!5!white}%lightgrey}

\tcbset{size=fbox}%,colframe=black,colback=white,arc=2pt}%boxsep=0pt,left=4pt,right=4pt,top=2.5pt,bottom=2.5pt}

\begin{document}

\title{An Introduction to Proof via \\Inquiry-Based Learning}
\author{Dana C.~Ernst, PhD\\
Northern Arizona University}
\date{Version Spring 2025}

\maketitle
\thispagestyle{empty}

\noindent\copyright{ \the\year\ Dana C.~Ernst.  Some Rights Reserved.\\

\bigskip

\noindent This work is licensed under the Creative Commons Attribution-Share Alike 4.0 International License.  You may copy, distribute, display, and perform this copyrighted work, but only if you give credit to Dana C.~Ernst, and all derivative works based upon it must be published under the Creative Commons Attribution-Share Alike 4.0 International License. Please attribute this work to Dana C.~Ernst, Mathematics Faculty at Northern Arizona University, \url{dana.ernst@nau.edu}, as well as the individuals listed in the Acknowledgements. To view a copy of this license, visit
\begin{center}
\url{https://creativecommons.org/licenses/by-sa/4.0/}
\end{center}
or send a letter to Creative Commons, 171 Second Street, Suite 300, San Francisco, California, 94105, USA.}

\medskip

\begin{center}
\ccbysa
\end{center}

\bigskip

\tableofcontents
%\thispagestyle{empty}

\include{Preface}
\chapter*{Acknowledgements}\label{chap:acknowledgements}
\addcontentsline{toc}{chapter}{\protect\numberline{}Acknowledgements}

The first draft of this book was written in 2009. At that time, several of the sections were adaptations of course materials written by Matthew Jones (CSU Dominguez Hills) and Stan Yoshinobu (University of Toronto). The current version of the book is the result of many iterations that involved the addition of new material, retooling of existing sections, and feedback from instructors that have used the book. The current version of the book is a far cry from what it looked like in 2009.

This book has been an open-source project since day one. Instructors and students can download the PDF for free and modify the source as they see fit. Several instructors and students have provided extremely useful feedback, which has improved the book at each iteration. Moreover, due to the open-source nature of the book, I have been able to incorporate content written by others. Below is a partial list of people (alphabetical by last name) that have contributed content, advice, or feedback.

\begin{itemize}
\item \href{https://math.depaul.edu/cdrupies/}{Chris Drupieski}, \href{http://math.depaul.edu/tpeter21/}{T.~Kyle Petersen}, and \href{http://math.depaul.edu/bridget/}{Bridget Tenner} (DePaul University). Modifications that these three made to the book inspired me to streamline some of the exposition, especially in the early chapters.
\item \href{http://www.paulellis.org}{Paul Ellis} (Manhattanville College). Paul has provided lots of useful feedback and several suggestions for improvements. Paul suggested problems for Chapter~\ref{chap:Induction} and provided an initial draft of Section~\ref{sec:Images_and_Preimages}: Images and Preimages of Functions.
\item \href{http://jasongrout.org}{Jason Grout} (Bloomberg, L.P.).  I am extremely grateful to Jason for feedback on early versions of this manuscript, as well as for helping me with a variety of technical aspects of writing an open-source textbook.
\item \href{https://www.linkedin.com/in/andershendrickson/}{Anders Hendrickson} (Milliman). Anders is the original author of the content in Appendix~\ref{appendix:elements_of_style}: Elements of Style for Proofs. The current version in Appendix~\ref{appendix:elements_of_style} is a result of modifications made by myself with some suggestions from David Richeson.
\item \href{http://www.hsc.edu/rebecca-jayne}{Rebecca Jayne} (Hampden--Sydney College). The current version of Section~\ref{sec:CompleteInduction}: Complete Induction is a derivative of content originally contributed by Rebecca.
\item \href{https://www.linkedin.com/in/matt-jones-a704aab/}{Matthew Jones} (CSU Dominguez Hills) and \href{https://www.math.toronto.edu/cms/people/faculty/yoshinobu-stan/}{Stan Yoshinobu} (University of Toronto). A few of the sections were originally adaptations of notes written by Matt and Stan. Early versions of this textbook relied heavily on their work. Moreover, Matt and Stan were two of the key players that contributed to shaping my approach to teaching.
\item \href{https://divisbyzero.com}{David Richeson} (Dickinson College). David is responsible for much of the content in Appendix~\ref{appendix:fancy_math_terms}: Fancy Mathematical Terms, Appendix~\ref{appendix:paradoxes}: Paradoxes, and Appendix~\ref{appendix:definitions}: Definitions in Mathematics. In addition, the current version of Chapter~\ref{chap:ThreeFamousTheorems}: Three Famous Theorems is heavily based on content contributed by David.
\item \href{http://www2.kenyon.edu/Depts/Math/schumacherc/public_html/}{Carol Schumacher} (Kenyon College). When I was transitioning to an IBL approach to teaching, Carol was one of my mentors and played a significant role in my development as a teacher.  Moreover, this work is undoubtedly influenced by Carol's excellent book \emph{Chapter Zero: Fundamental Notions of Advanced Mathematics}, which I used when teaching my very first IBL course.
\item \href{http://webpages.csus.edu/wiscons/}{Josh Wiscons} (CSU Sacramento). The current version of Section~\ref{sec:ModularArithmetic}: Modular Arithmetic is a derivative of content contributed by Josh.
\end{itemize}

\include{Introduction}
\include*{IntroToMath}
\include*{TasteNumberTheory}
\include*{IntroToLogic}
\include*{NegatingAndContradiction}
\include*{IntroQuantification}
\include*{MoreQuantification}
\include*{SetTheory}
\include*{Sets}
\include*{RussellsParadox}
\include*{PowerSets}
\include*{IndexingSets}
\include*{CartesianProducts}
\include*{Induction}
\include*{IntroInduction}
\include*{MoreInduction}
\include*{CompleteInduction}
\include*{WellOrderingPrinciple}
\include*{RealNumbers}
\include*{AxiomsRealNumbers}
\include*{Topology}
\include*{ThreeFamousTheorems}
\include*{FundamentalTheoremArithmetic}
\include*{IrrationalityRoot2}
\include*{InfinitudeOfPrimes}
\include*{RelationsPartitions}
\include*{Relations}
\include*{EquivalenceRelations}
\include*{Partitions}
\include*{ModularArithmetic}
%\include*{Posets}
\include*{Functions}
\include*{IntroFunctions}
\include*{InjectiveSurjectiveFunctions} 
\include*{CompositionsInverses}
\include*{ImagesInverseImages}
\include*{ContinuousFunctions}
\include*{Cardinality}
\include*{IntroCardinality}
\include*{FiniteSets}
\include*{InfiniteSets}
\include*{CountableSets}
\include*{UncountableSets}
%Appendices
\include{ElementsOfStyle}
\chapter{Fancy Mathematical Terms}
\label{appendix:fancy_math_terms}

Here are some important mathematical terms that you will encounter throughout mathematics.

\begin{enumerate}
\item \textbf{Definition}---a precise and unambiguous description of the meaning of a mathematical term.  It characterizes the meaning of a word by giving all the properties and only those properties that must be true.
\item \textbf{Theorem}---a mathematical statement that is proved using rigorous mathematical reasoning.  In a mathematical paper, the term theorem is often reserved for the most important results.
\item \textbf{Proposition}---a proved and often interesting result, but generally less important than a theorem. Alternatively, a \textbf{proposition} may refer to a sentence that is either true or false but never both (see Definition~\ref{def:proposition}). 
\item \textbf{Lemma}---a minor result whose sole purpose is to help in proving a theorem.  It is a stepping stone on the path to proving a theorem. Occasionally lemmas can take on a life of their own (Zorn's Lemma, Urysohn's Lemma, Burnside's Lemma, Sperner's Lemma).
\item \textbf{Corollary}---a result in which the (usually short) proof relies heavily on a given theorem (we often say that ``this is a corollary of Theorem A'').
\item \textbf{Conjecture}---a statement that is unproved, but is believed to be true (Collatz Conjecture, Goldbach Conjecture, Twin prime Conjecture).
\item \textbf{Claim}---an assertion that is then proved.  It is often used like an informal lemma.
\item \textbf{Counterexample}---a specific example showing that a statement is false.
\item \textbf{Axiom/Postulate}---a statement that is assumed to be true without proof. These are the basic building blocks from which all theorems are proved (Euclid's five postulates, axioms of ZFC, Peano axioms).
\item \textbf{Identity}---a mathematical expression giving the equality of two (often variable) quantities (trigonometric identities, Euler's identity).
\item \textbf{Paradox}---a statement that can be shown, using a given set of axioms and definitions, to be both true and false. Paradoxes are often used to show the inconsistencies in a flawed axiomatic theory (e.g., Russell's Paradox).  The term paradox is also used informally to describe a surprising or counterintuitive result that follows from a given set of rules (Banach-Tarski Paradox, Alabama Paradox, Gabriel's Horn).
\end{enumerate}
\include{Paradoxes}
\chapter{Definitions in Mathematics}
\label{appendix:definitions}

It is difficult to overstate the importance of definitions in mathematics. Definitions play a different role in mathematics than they do in everyday life. 

Suppose you give your friend a piece of paper containing the definition of the rarely-used word \textbf{rodomontade}. According to the Oxford English Dictionary\footnote{http://www.oed.com/view/Entry/166837} (OED) it is:
\begin{quote}
A vainglorious brag or boast; an extravagantly boastful, arrogant, or bombastic speech or piece of writing; an arrogant act.
\end{quote}
Give your friend some time to study the definition. Then take away the paper. Ten minutes later ask her to define rodomontade. Most likely she will be able to give a reasonably accurate definition. Maybe she'd say something like, ``It is a speech or act or piece of writing created by a pompous or egotistical person who wants to show off how great they are.'' It is unlikely that she will have quoted the OED word-for-word. In everyday English that is fine---you would probably agree that your friend knows the meaning of the rodomontade. This is because most definitions are \emph{descriptive}. They describe the common usage of a word. 

Let us take a mathematical example. The OED\footnote{http://www.oed.com/view/Entry/40280}  gives this definition of \textbf{continuous}.
\begin{quote}
Characterized by continuity; extending in space without interruption of substance; having no interstices or breaks; having its parts in immediate connection; connected, unbroken.
\end{quote}
Likewise, we often hear calculus students speak of a continuous function as one whose graph can be drawn ``without picking up the pencil.'' This definition is descriptive. However, as we learned in calculus, the picking-up-the-pencil description is not a perfect description of continuous functions. This is not a mathematical definition. 

Mathematical definitions are \emph{prescriptive}. The definition must prescribe the exact and correct meaning  of a word. Contrast the OED's descriptive definition of continuous with the definition of continuous found in a real analysis textbook.
\begin{quote}
A function $f:A\to \mathbb{R}$ is \textbf{continuous at a point} $c\in A$ if,  for all $\varepsilon>0$, there exists $\delta>0$ such that whenever $|x-c|<\delta$ (and $x\in A$) it follows that $|f(x)-f(c)|<\varepsilon$. If $f$ is continuous at every point in the domain $A$, then we say that $f$ is \textbf{continuous on} $A$.\footnote{This definition is taken from page 109 of Stephen Abbott's \emph{Understanding Analysis}, but the definition would be essentially the same in any modern real analysis textbook.} 
\end{quote}
In mathematics there is very little freedom in definitions. Mathematics is a deductive theory; it is impossible to state and prove theorems without clear definitions of the mathematical terms. The definition of a term must completely, accurately, and unambiguously describe the term. Each word is chosen very carefully and the order of the words is  critical. In the definition of continuity changing ``there exists'' to ``for all,'' changing the orders of quantifiers, changing $<$ to $\leq$ or $>$, or changing $\mathbb{R}$ to $\mathbb{Z}$ would completely change the meaning of the definition. 

What does this mean for you, the student? Our recommendation is that at this stage you memorize the definitions word-for-word. It is the safest way to guarantee that you have it correct. As you gain confidence and familiarity with the subject you may be ready to modify the wording. You may want to change ``for all'' to ``given any'' or you may want to change $|x-c|<\delta$ to $-\delta<x-c<\delta$ or to ``the distance between $x$ and $c$ is less than $\delta$.'' 

Of course, memorization is not enough; you must have a conceptual understanding of the term, you must see how the formal definition matches up with your conceptual understanding, and you must know how to work with the definition. It is perhaps with the first of these that descriptive definitions are useful. They are useful for building intuition and for painting the ``big picture.'' Only after days (weeks, months, years?) of experience does one get an intuitive feel for the epsilon-delta definition of continuity; most mathematicians have the ``picking-up-the-pencil'' definitions in their head. This is fine as long as we know that it is imperfect, and that when we prove theorems about continuous functions in mathematics we use the mathematical definition. 

We end this discussion with an amusing real-life example in which a descriptive definition was not sufficient. In 2003 the German version of the game show \emph{Who Wants to Be a Millionaire?} contained the following question: ``Every rectangle is: (a) a rhombus, (b) a trapezoid, (c) a square, (d) a parallelogram.'' 

The confused contestant decided to skip the question and left with \euro 4000. Afterward the show received letters from irate viewers. Why were the contestant and the viewers upset with this problem? Clearly a rectangle is a parallelogram, so (d) is the answer. But what about (b)? Is a rectangle a trapezoid? We would describe a trapezoid as a quadrilateral with a pair of parallel sides. But this leaves open the question: can a trapezoid have \emph{two} pairs of parallel sides or must there only be \emph{one} pair? The viewers said two pairs is allowed, the producers of the television show said it is not. This is a case in which a clear, precise, mathematical definition is required.

\end{document}
