\documentclass[12pt,oneside]{book}

\usepackage[scale=2]{ccicons}
\usepackage{caption}
\usepackage[labelfont={}]{subcaption}
\captionsetup{subrefformat=parens}
\usepackage{booktabs}
\usepackage{epigraph}
\usepackage{enumitem}
\usepackage{newclude}
\usepackage{multicol}
\usepackage{tabu}
\usepackage[table]{xcolor}
\usepackage{tikz}
\usetikzlibrary{arrows,automata,positioning,fit,shapes}
\usetikzlibrary{decorations.markings}
\usetikzlibrary{decorations.pathreplacing}
\usepackage{rotating}
\usepackage[notextcomp]{kpfonts} 
\usepackage{graphicx}
\usepackage{eurosym}
\usepackage{amsfonts}
\usepackage{mathtools}
\usepackage{amsmath}
\usepackage{amssymb}
\usepackage{stmaryrd}
\usepackage{wasysym}
\usepackage{amsthm}
\usepackage[margin=1in]{geometry}
\usepackage[hang,flushmargin]{footmisc}
\usepackage{color}
\usepackage[breaklinks]{hyperref}
\hypersetup{
	colorlinks=true,
	linkcolor=darkblue,
	anchorcolor=darkblue,
	citecolor=darkblue,
	pagecolor=darkblue,
	urlcolor=darkblue,
	pdftitle={},
	pdfauthor={},
    bookmarksnumbered
}

%%% Colors %%%

\definecolor{darkblue}{rgb}{0, 0, .6}
\definecolor{grey}{rgb}{.7, .7, .7}
\definecolor{naugreen}{cmyk}{.43,0,.34,.38}
\definecolor{naublue}{cmyk}{1,.72,0,.32}
\definecolor{mediterranean}{cmyk}{.67,0,.08,.3}
\definecolor{rose}{cmyk}{0,1.00,.20,0}
\definecolor{darkorchid}{cmyk}{.6,.9,0,.05}
\definecolor{butterfly}{cmyk}{.95,.59,0,.10}
\definecolor{springgreen}{cmyk}{1.00,0,.70,.02}
\definecolor{darkred}{cmyk}{0,1,1,.5}
\definecolor{nectarine}{cmyk}{0,0.70,1.00,0}
\definecolor{icyblue}{cmyk}{.84,.25,0,.06}
\definecolor{orange}{RGB}{255,102,0}
\definecolor{ggreen}{RGB}{0,153,0}
\definecolor{darkblue}{RGB}{0,0,255}
\definecolor{purple}{RGB}{153,51,255}
\definecolor{turq}{RGB}{72,209,204}
\definecolor{lightskyblue}{cmyk}{.4,.11,0,.2}
\definecolor{softplum}{cmyk}{.02,.45,0,.68}%.37
\definecolor{lightorange}{cmyk}{0,.45,.59,.02}
\definecolor{brickorange}{cmyk}{0,.59,.59,.20}
\definecolor{deeppink}{cmyk}{0,.6,.3,.3}
\definecolor{orchid}{cmyk}{0,.49,.02,.05}

\usepackage{fancyhdr}
\pagestyle{fancy}
\lhead{\leftmark}
\chead{}
\rhead{}
\lfoot{}
\cfoot{\thepage}
\rfoot{}

\theoremstyle{definition}
\newtheorem{theorem}{Theorem}[chapter]
\newtheorem{acknowledgement}[theorem]{Acknowledgement}
\newtheorem{algorithm}[theorem]{Algorithm}
\newtheorem{axiom}[theorem]{Axiom}
\newtheorem{case}[theorem]{Case}
\newtheorem{claim}[theorem]{Claim}
\newtheorem{conclusion}[theorem]{Conclusion}
\newtheorem{condition}[theorem]{Condition}
\newtheorem{conjecture}[theorem]{Conjecture}
\newtheorem{corollary}[theorem]{Corollary}
\newtheorem{criterion}[theorem]{Criterion}
\newtheorem{definition}[theorem]{Definition}
\newtheorem{example}[theorem]{Example}
\newtheorem{exercise}[theorem]{Exercise}
\newtheorem{journal}[theorem]{Journal}
\newtheorem{lemma}[theorem]{Lemma}
\newtheorem{notation}[theorem]{Notation}
\newtheorem{problem}[theorem]{Problem}
\newtheorem{proposition}[theorem]{Proposition}
\newtheorem{remark}[theorem]{Remark}
\newtheorem{solution}[theorem]{Solution}
\newtheorem{summary}[theorem]{Summary}
\newtheorem{skeleton}[theorem]{Skeleton Proof}
\newtheorem{activity}[theorem]{Activity}
\newtheorem{intuitivedef}[theorem]{Intuitive Definition}
\newtheorem{question}[theorem]{Question}
\newtheorem{fieldaxioms}[theorem]{Field Axioms}
\newtheorem{orderaxioms}[theorem]{Order Axioms}
\newtheorem{completenessaxiom}[theorem]{Completeness Axiom}
\newtheorem{additionalaxioms}[theorem]{Additional Order Axioms}

\newsavebox{\savepar}
\newenvironment{textbox}{\noindent\begin{lrbox}{\savepar}\begin{minipage}[c]{.98\textwidth}}{\end{minipage}\end{lrbox}\fcolorbox{black}{white}{\usebox{\savepar}}}

\newcommand{\dom}{\operatorname{Dom}}
\newcommand{\codom}{\operatorname{Codom}}
\newcommand{\range}{\operatorname{Rng}}
\newcommand{\lcm}{\operatorname{lcm}}
\newcommand{\card}{\operatorname{card}}
\newcommand{\rel}{\operatorname{rel}}
\newcommand{\Rel}{\operatorname{Rel}}

\begin{document}

\title{An Introduction to Proof via \\Inquiry-Based Learning}
\author{Dana C.~Ernst, PhD\\
Northern Arizona University}
\date{Version Fall 2021}

\maketitle
\thispagestyle{empty}

\noindent\copyright{ \the\year\ Dana C.~Ernst.  Some Rights Reserved.\\

\bigskip

\noindent This book is intended to be a task sequence for an introduction to proof course that utilizes an inquiry-based learning (IBL) approach.  You can find the most up-to-date version of this textbook on GitHub:
\begin{center}
\url{http://dcernst.github.io/IBL-IntroToProof/}
\end{center}
I would be thrilled if you used this textbook and improved it. If you make any modifications, you can either make a pull request on GitHub or submit the improvements via email.  You are also welcome to fork the source and modify the text for your purposes as long as you maintain the license below.

\bigskip

\noindent This work is licensed under the Creative Commons Attribution-Share Alike 4.0 United States License.  You may copy, distribute, display, and perform this copyrighted work, but only if you give credit to Dana C.~Ernst, and all derivative works based upon it must be published under the Creative Commons Attribution-Share Alike 4.0 International License. Please attribute this work to Dana C.~Ernst, Mathematics Faculty at Northern Arizona University, \url{dana.ernst@nau.edu}, as well as the individuals listed below. To view a copy of this license, visit
\begin{center}
\url{https://creativecommons.org/licenses/by-sa/4.0/}
\end{center}
or send a letter to Creative Commons, 171 Second Street, Suite 300, San Francisco, California, 94105, USA.}

\medskip

\begin{center}
\ccbysa
\end{center}

\bigskip

\tableofcontents
%\thispagestyle{empty}

\chapter*{Preface}
\addcontentsline{toc}{chapter}{\protect\numberline{}Preface}


When I started teaching, I mimicked the experiences I had as a student. Because it was all I knew, I lectured. By standard metrics, this seemed to work out just fine. Glowing student and peer evaluations, as well as reoccurring teaching awards, indicated that I was effectively doing my job. People consistently told me that I was an excellent teacher. However, two observations made me reconsider how well I was really doing. Namely, many of my students seemed to depend on me to be successful, and second, they retained only some of what I had taught them. In the words of Dylan Retsek (Cal Poly):
``Things my students claim that I taught them masterfully, they don't know."
Inspired by a desire to address these concerns, I began transitioning away from direct instruction towards a more student-centered approach. The goals and philosophy behind inquiry-based learning (IBL) resonate deeply with my ideals, which is why I have embraced this paradigm. IBL is a pedagogical framework that I will elaborate on in the next section.


My primary objective is to play a role in developing persistent problem solvers...learn how to learn...content secondary...educated citizen...

provide students with the opportunity 


Mathematics is not about calculations, but ideas. My goal as an instructor is to provide students with the opportunity to grapple with these ideas and to be immersed in the process of mathematical discovery. Repeatedly engaging in this process hones the mind and develops mental maturity marked by clear and rigorous thinking. Bertrand Russell wrote that
``Mathematics, rightly viewed, possesses not only truth, but supreme beauty."
REWRITE THIS PART: Like music and art, mathematics provides an opportunity for enrichment. The medium of a painter is color and shape, whereas the medium of a mathematician is abstract thought. As a teacher, I attempt to convey the elegance and aesthetic value of mathematics, and I regularly remind my students that the creative aspect of mathematics is what captivates me and fuels my motivation to keep learning and exploring.

While the content we teach our students is important, it is not enough. What is perhaps more true than ever, is that we need individuals capable of asking and exploring questions in contexts that do not yet exist and to be able to tackle problems they have never encountered. It is important that we put these issues front and center and place an explicit focus on students producing, rather than consuming, knowledge. If we truly want our students to be independent, inquisitive, and persistent, then we need to provide them with the means to acquire these skills.


%shorter
This book is intended to be used for a one-semester introduction to proof course. The book is designed around an inquiry-based learning (IBL) approach to teaching and learning mathematics, but makes no assumptions about the specifics of how the instructor chooses to implement this approach.  Loosely speaking, IBL is a student-centered method of teaching that engages students in sense-making activities.  Students are given tasks requiring them to solve problems, conjecture, experiment, explore, create, and communicate.  Rather than showing facts or a clear, smooth path to a solution, the instructor guides and mentors students via well-crafted problems through an adventure in mathematical discovery. The book includes more content than one can expect to cover in a single semester. This allows the instructor to pick and choose the sections that suit their needs and desires. Each chapter takes a focused approach to the included topics, but also includes many gentle exercises aimed at developing intuition.


%longer
This book is intended to be used for a one-semester/quarter introduction to proof course (sometimes referred to as a transition to proof course). The intended audience is mathematics majors and minors. The book is designed around an inquiry-based learning (IBL) approach to teaching and learning mathematics, but makes no assumptions about the specifics of how the instructor chooses to implement this approach.  Loosely speaking, IBL is a student-centered method of teaching that engages students in sense-making activities.  Students are given tasks requiring them to solve problems, conjecture, experiment, explore, create, and communicate.  Rather than showing facts or a clear, smooth path to a solution, the instructor guides and mentors students via well-crafted problems through an adventure in mathematical discovery.  According to \href{https://www.colorado.edu/eer/sites/default/files/attached-files/laursenrasmussencommentaryauthorversion0219.pdf}{Laursen and Rasmussen (2019)}, the Four Pillars of IBL are:
\begin{itemize}
\item Students engage deeply with coherent and meaningful mathematical tasks.
\item Students collaboratively process mathematical ideas.
\item Instructors inquire into student thinking.
\item Instructors foster equity in their design and facilitation choices.
\end{itemize}

The book includes more content than one can expect to cover in a single semester. This allows the instructor to pick and choose the sections that suit their needs and desires. Each chapter takes a focused approach to the included topics, but also includes many gentle exercises aimed at developing intuition.




%%%%%%%%%

You are the creators. This book is a guide.

This book will not show you how to solve all the problems that are presented, but it should \emph{enable} you to find solutions, on your own and working together. The material you are about to study did not come together fully formed at a single moment in history. It was composed gradually over the course of centuries, with various mathematicians building on the work of others, improving the subject while increasing its breadth and depth.

Mathematics is essentially a human endeavor. Whatever you may believe about the true nature of mathematics---does it exist eternally in a transcendent Platonic realm, or is it contingent upon our shared human consciousness?---our \emph{experience} of mathematics is temporal, personal, and communal. Like music, mathematics that is encountered only as symbols on a page remains inert. Like music, mathematics must be created in the moment, and it takes time and practice to master each piece. The creation of mathematics takes place in writing, in conversations, in explanations, and most profoundly in the mental construction of its edifices on the basis of reason and observation.

To continue the musical analogy, you might think of this textbook like a performer's score. Much is included to direct you towards particular ideas, but much is missing that can only be supplied by you: participation in the creative process that will make those ideas come alive. Moreover, your success will depend on the pursuit of both \emph{individual} excellence and \emph{collective} achievement. Like a musician in an orchestra, you should bring your best work and be prepared to blend it with others' contributions.

In any act of creation, there must be room for experimentation, and thus allowance for mistakes, even failure. A key goal of our community is that we support each other---sharpening each other's thinking but also bolstering each other's confidence---so that we can make failure a \emph{productive} experience. Mistakes are inevitable, and they should not be an obstacle to further progress. It is normal to struggle and be confused as you work through new material. Accepting that means you can keep working even while feeling stuck, until you overcome and reach even greater accomplishments.

This book is a guide. You are the creators.
\chapter*{Acknowledgements}
\addcontentsline{toc}{chapter}{\protect\numberline{}Acknowledgements}

The first draft of this book was written in 2009. At that time, several of the sections were adaptations of course materials written by Matthew Jones (CSU Dominguez Hills) and Stan Yoshinobu (University of Toronto). The current version of the book is the result of many iterations that involved the addition of new material, retooling of existing sections, and feedback from instructors that have used the book. The current version of the book is a far cry from what it looked like in 2009.

This book has been an open-source project since day one. Instructors and students can download the PDF for free and modify the source as they see fit. Several instructors and students have provided extremely useful feedback, which has improved the book at each iteration. Moreover, due to the open-source nature of the book, I have been able to incorporate content written by others. Below is a partial list of people (alphabetical by last name) that have contributed content, advice, or feedback.

\begin{itemize}
\item \href{https://math.depaul.edu/cdrupies/}{Chris Drupieski}, \href{http://math.depaul.edu/tpeter21/}{T.~Kyle Petersen}, and \href{http://math.depaul.edu/bridget/}{Bridget Tenner} (DePaul University). Modifications that these three made to the book inspired me to streamline some of the exposition, especially in the early chapters.
\item \href{http://www.paulellis.org}{Paul Ellis} (Manhattanville College). Paul has provided lots of useful feedback and several suggestions for improvements. Paul suggested problems for Chapter~\ref{chap:Induction} and provided an initial draft of Section~\ref{sec:Images and Preimages}: Images and Preimages of Functions.
\item \href{http://jasongrout.org}{Jason Grout} (Bloomberg, L.P.).  I am extremely grateful to Jason for feedback on early versions of this manuscript, as well as for helping me with a variety of technical aspects of writing an open-source textbook.
\item \href{https://www.linkedin.com/in/andershendrickson/}{Anders Hendrickson} (Milliman). Anders is the original author of the content in Appendix~\ref{appendix:elements_of_style}: Elements of Style for Proofs. The current version in Appendix~\ref{appendix:elements_of_style} is a result of modifications made by myself with some suggestions from David Richeson.
\item \href{http://www.hsc.edu/rebecca-jayne}{Rebecca Jayne} (Hampden--Sydney College). The current version of Section~\ref{sec:CompleteInduction}: Complete Induction is a derivative of content originally contributed by Rebecca.
\item \href{https://www.linkedin.com/in/matt-jones-a704aab/}{Matthew Jones} (CSU Dominguez Hills) and \href{https://www.math.toronto.edu/cms/people/faculty/yoshinobu-stan/}{Stan Yoshinobu} (University of Toronto). A few of the sections were originally adaptations of notes written by Matt and Stan. Early versions of this textbook relied heavily on their work. Moreover, Matt and Stan were two of the key players that contributed to shaping my approach to teaching.
\item \href{https://divisbyzero.com}{David Richeson} (Dickinson College). David is responsible for much of the content in Appendix~\ref{appendix:fancy_math_terms}: Fancy Mathematical Terms, Appendix~\ref{appendix:paradoxes}: Paradoxes, and Appendix~\ref{appendix:definitions}: Definitions in Mathematics. In addition, the current version of Chapter~\ref{chap:ThreeFamousTheorems}: Three Famous Theorems is heavily based on content contributed by David.
\item \href{http://www2.kenyon.edu/Depts/Math/schumacherc/public_html/}{Carol Schumacher} (Kenyon College). When I was transitioning to an IBL approach to teaching, Carol was one of my mentors and played a significant role in my development as a teacher.  Moreover, this work is undoubtably influenced my Carol's excellent book \emph{Chapter Zero: Fundamental Notions of Advanced Mathematics}, which I used when teaching my very first IBL course.
\item \href{http://webpages.csus.edu/wiscons/}{Josh Wiscons} (CSU Sacramento). The current version of Section~\ref{sec:ModularArithmetic}: Modular Arithmetic is a derivative of content contributed by Josh.
\end{itemize}

\chapter{Introduction}\label{chap:intro}

\epigraphhead[70pt]{
\epigraph{The mathematician does not study pure mathematics because it is useful; he studies it because he delights in it, and he delights in it because it is beautiful.}{Henri Poincar\'e, mathematician \& physicist}}

\begin{section}{What is This Book All About?}%new title?

The foundations of mathematics refers to logic and set theory; the axioms of number and space.  Also, it refers to an introduction to the techniques of proof, and at a larger level the process of \emph{doing mathematics}.  Proof is central to doing mathematics.

Up to this point, it is likely that your experience of mathematics has been about using formulas and algorithms. That is only one part of mathematics. Mathematicians do much more than just use formulas.  Mathematicians experiment, make conjectures, write definitions, and prove theorems.  While engaging with the material contained in this book, we will learn about doing all of these things. The focus of this book is on learning to \emph{do} mathematics. 
  
Just like learning to play an instrument or sport, you will have to learn new skills and ideas.  Sometimes you will feel good, sometimes frustrated.  You will probably go through a range of feelings from being exhilarated, to being stuck.  Figuring it out, victories, defeats, and all that is part of real life is what you can expect.  Most importantly it will be rewarding.  Learning mathematics requires dedication.  It will require that you be patient despite periods of confusion.  It will require that you persevere in order to understand.  As the author of this text, I am here to guide you, but I cannot do the learning for you, just as a music teacher cannot move your fingers and your heart for you.  Only you can do that.  Do your best, be prepared to put in a lot of time, and do all the work. When you work hard and you come to understand, you feel good about yourself.  In the meantime, you have to believe that your work will pay off in intellectual development.

Your progress will be fueled by your ability to wrestle with mathematical ideas and to prove theorems.  As you work through the book, you will find that you have ideas for proofs, but you are unsure of them.  Do not be afraid to tinker and make mistakes.  You can always revisit your work as you become more proficient. Do not expect to do most things perfectly on your first attempt. 

This is a very exciting time in your mathematical career.  It is where you learn what mathematics is really about!

\epigraph{Mathematics, rightly viewed, possesses not only truth, but supreme beauty---a beauty cold and austere, like that of sculpture, without appeal to any part of our weaker nature, without the gorgeous trappings of painting or music, yet sublimely pure, and capable of a stern perfection such as only the greatest art can show. The true spirit of delight, the exaltation, the sense of being more than Man, which is the touchstone of the highest excellence, is to be found in mathematics as surely as poetry.}{Bertrand Russell, philosopher \& mathematician}

\end{section}

\begin{section}{An Inquiry-Based Approach}

This is not a typical textbook or one in which mimicking prefabricated examples will lead you to success. You will be expected to work actively to construct your own understanding of the topics at hand. Many of the concepts you learn and problems you work on will be new to you and ask you to stretch your thinking. You will experience \emph{frustration} and \emph{failure} before you experience \emph{understanding}. This is part of the normal learning process. If you are doing things well, you should be confused on a regular basis. The material is too rich for a human being to completely understand it immediately. Your viability as a professional in the modern workforce depends on your ability to embrace this learning process and make it work for you.

%In order to promote a more active participation in your learning, this book adheres to an educational philosophy called inquiry-based learning (IBL).  Loosely speaking, IBL is a student-centered method of teaching mathematics that engages students in sense-making activities.  Students are given tasks requiring them to solve problems, conjecture, experiment, explore, create, and communicate.  Rather than showing facts or a clear, smooth path to a solution, this book will guide and mentor you via well-crafted problems through an adventure in mathematical discovery. 

In many mathematics classrooms, ``doing mathematics" means following the rules dictated by the teacher, and ``knowing mathematics" means remembering and applying them. However, an inquiry-based-learning (IBL) approach challenges students to create or discover mathematics. IBL is a student-centered method of teaching mathematics that engages students in sense-making activities.  Students are given tasks requiring them to solve problems, conjecture, experiment, explore, create, and communicate.  Rather than showing facts or a clear, smooth path to a solution, this book will guide and mentor you via well-crafted problems through an adventure in mathematical discovery. Effective IBL courses come in a variety of forms, but they all possess a few essential ingredients. According to \href{https://www.colorado.edu/eer/sites/default/files/attached-files/laursenrasmussencommentaryauthorversion0219.pdf}{Laursen and Rasmussen (2019)}, the Four Pillars of IBL are:
\begin{itemize}
\item Students engage deeply with coherent and meaningful mathematical tasks.
\item Students collaboratively process mathematical ideas.
\item Instructors inquire into student thinking.
\item Instructors foster equity in their design and facilitation choices.
\end{itemize}

Moreover, students should---as much as possible---be responsible for guiding the acquisition of knowledge and validating the ideas presented. That is, students should not be looking to the instructor as the sole authority. In an IBL course, instructor and students have joint responsibility for the depth and progress of the course.

\textbf{Is this from Joshua Bowman???}
%%%
While progressing through this book, you will be asked to engage with the material and to produce mathematics. The best way to learn mathematics is by doing mathematics.  Someone cannot master a musical instrument or a martial art by simply watching, and in a similar fashion, you cannot master mathematics by simply watching; you must do mathematics!

In any act of creation, there must be room for experimentation, and thus allowance for mistakes, even failure. Mistakes are inevitable, and they should not be an obstacle to further progress. Productive struggle and mistakes provide opportunities for growth.  It is normal to struggle and be confused as you work through new material. Accepting that means you can keep working even while feeling stuck, until you overcome and reach even greater accomplishments.
%%%%

Furthermore, it is important to understand that solving genuine problems is difficult and takes time.  You should not expect to complete each problem in 10 minutes or less.  Sometimes you might have to stare at the problem for an hour before even understanding how to get started.

%from Kyle: I like to think of the difference between a lecture-driven classroom and an IBL classroom like the difference between taking a cable car to the summit of a mountain and climbing to the top of the same mountain. One could argue that both approaches can get you to the top of the mountain, but clearly the experiences are different. And when you come to a new mountain, for which there is no cable car, who is better prepared to reach the summit?

%also think about: Learning like a jungle tiger.

\textbf{Should this go in Instructor Guide???}

Evidence in favor of some form of active engagement of students is strong across STEM disciplines. \href{???}{Freeman et al.~(2014)} conducted a meta-analysis of 225 studies of various forms of active learning, and found that students were 1.5 times more likely to fail in traditional courses as compared to active learning courses, and students in active learning courses outperformed students in traditional courses by 0.47 standard deviations on examinations and concept inventories. The following snippet from Freeman et al.~(2014) captures the importance of utilizing active learning across STEM education:
\begin{quote}
\emph{``The results raise questions about the continued use of traditional lecturing as a control in research studies, and support active learning as the preferred, empirically validated teaching practice in regular classrooms."}
\end{quote}

For IBL specifically, a research group from the University of Colorado Boulder led by Sandra Laursen conducted a comprehensive study of student outcomes in IBL undergraduate mathematics courses while linking these outcomes to students' and instructors' experiences of IBL (see Laursen et al. 2011; Laursen 2013; Kogan and Laursen 2014; Laursen et al. 2014). This quasi-experimental, longitudinal study examined over 100 courses at four different campuses over a period that spanned two years.

On average over 60\% of IBL class time was spent on student-centered activities including student-led presentations, discussion, and small-group work. In contrast, in non-IBL courses, 87\% of class time was devoted to students' listening to an instructor talk. In addition, the IBL sections were rated more highly for a supportive classroom environment and students conveyed that engaging in meaningful mathematical tasks while collaborating was essential to their learning. Below is a brief summary of some of the outcomes of Laursen et al.'s work.
\begin{itemize}
\item After an IBL or comparative course, IBL students reported higher learning gains than their non-IBL peers, across cognitive, affective, and collaborative domains of learning.
\item In later courses, students who had taken an IBL course earned grades as good or better than those of students who took non-IBL sections, despite having ``covered" less material.
\item Non-IBL courses show a marked gender gap: across the board, women reported lower learning gains and less supportive attitudes than did men (effect size 0.4--0.5). Women's confidence and sense of mastery of mathematics, and their interest in continued study of math were lower. This difference appears to be primarily affective, not due to real differences in women's mathematical preparation or achievement.
\item This gender gap was erased in IBL classes: women's learning gains were equal to men's, and their confidence and intent to persist similar. IBL approaches leveled the playing field for women, fixing a course that is problematic for women yet with no harm to men.
\end{itemize}

\epigraph{Don't fear failure.  Not failure, but low aim, is the crime. In great attempts it is glorious even to fail.}{Bruce Lee, martial artist \& actor}

\end{section}

\begin{section}{Rights of the Learner}\label{sec:Rights of the Learner}
As a reader of this textbook, you have the right to:
\begin{enumerate}
\item be confused,
\item make a mistake and to revise your thinking,
\item speak, listen, and be heard, and
\item enjoy doing mathematics.
\end{enumerate}

\epigraph{You may encounter many defeats, but you must not be defeated.}{Maya Angelou, poet \& activist}

\end{section}

\begin{section}{Structure of the Textbook}

As you read the textbook, you will be required to digest the material in a meaningful way.  It is your responsibility to read and understand new definitions and their related concepts.  In addition, you will be asked to complete problems aimed at solidifying your understanding of the material.  Most importantly, you will be asked to make conjectures, produce counterexamples, and prove theorems. All of these tasks will almost always be challenging.

The items labeled as \textbf{Definition} and \textbf{Example} are meant to be read and digested.  However, the items labeled as \textbf{Problem}, \textbf{Theorem}, \textbf{Corollary}, and \textbf{Lemma} require action on your part.  Items labeled as \textbf{Problem} are sort of a mixed bag. Some Problems are computational in nature and aimed at improving your understanding of a particular concept while others ask you to provide a counterexample for a statement if it is false or to provide a proof if the statement is true. Items with the \textbf{Theorem}, \textbf{Corollary}, and \textbf{Lemma} designation are mathematical facts and the intention is for you to produce a valid proof of the given statement.  A lemma is a minor result whose sole purpose is to help in proving a theorem. The main difference between a theorem and a corollary is that corollaries are typically statements that follow quickly from a previous theorem.  In general, you should expect corollaries to have very short proofs.  However, that does not mean that you cannot produce a more lengthy yet valid proof of a corollary.

Oftentimes, the problems and theorems are guiding you towards a substantial, more  general result. Other times, they are designed to get you to apply ideas in a new way. One thing to always keep in mind is that every task in this book can be done by you, the student. But it may not be on your first try, or even your second.

Discussion of new topics is typically kept at a minimum and there are very few examples in this book. This is intentional.  One of the objectives of the items labeled as \textbf{Problem} is for you to produce the examples needed to internalize unfamiliar concepts.  The ultimate goal of this book is to help you develop a deep and meaningful understanding of the processes of producing mathematics by putting you in direct contact with mathematical phenomena.

\epigraph{Don't just read it; fight it! Ask your own questions, look for your own examples, discover your own proofs. Is the hypothesis necessary? Is the converse true? What happens in the classical special case? What about the degenerate cases? Where does the proof use the hypothesis?}{Paul Halmos, mathematician}

\end{section}

\begin{section}{Some Minimal Guidance}\label{sec:minimal guidance}
Especially in the opening sections, it will not be clear what facts from your prior experience in mathematics we are ``allowed" to use.  Unfortunately, addressing this issue is difficult and is something we will sort out along the way.  However, in general, here are some minimal guidelines to keep in mind.

\begin{itemize}
\item The statement you are proving should be on the same page as the beginning of your proof.   
\item You should indicate where the proof begins by writing ``\emph{Proof.}" at the beginning.  
\item Make it clear to yourself and the reader what your assumptions are at the very beginning of your proof. Typically, these statements will start off ``Assume\ldots", ``Suppose\ldots", or ``Let\ldots".  Sometimes there will be some implicit assumptions that we can omit, but at least in the beginning, you should get in the habit of clearly stating your assumptions up front. 
\item Carefully consider the order in which you write your proof. Loosely speaking, each sentence should follow from an earlier sentence in your proof or possibly a result you have already proved.
\item Unlike the experience many of you had writing proofs in your high school geometry class, our proofs should be written in complete sentences.  You should break sections of a proof into paragraphs and use proper grammar.  There are some pedantic conventions for doing this that will be pointed out along the way.  Initially, this will be an issue that you may struggle with, but you will get the hang of it.
\item There will be many situations where you will want to refer to an earlier definition, problem, theorem, corollary, or lemma.  In this case, you should reference the statement by number, but it is also helpful to the reader to summarize the statement you are citing.  For example, you might write something like, ``By Theorem~\ref{thm:two consecutive ints}, the sum of two consecutive integers is odd, and so\ldots" or ``By the definition of divides (Definition~\ref{def:divides}), it follows that\ldots". One thing worth pointing out is that if we are citing a definition, theorem, or problem by number, we should capitalize Definition, Theorem, or Problem, respectively (e.g., see examples above). Otherwise, we do not capitalize these words (e.g., ``By the definition of divides\ldots").
\item There will be times when we will need to do some basic algebraic manipulations.  You should feel free to do this whenever the need arises.  But you should show sufficient work along the way.  You do not need to write down justifications for basic algebraic manipulations (e.g., adding 1 to both sides of an equation, adding and subtracting the same amount on the same side of an equation, adding like terms, factoring, basic simplification, etc.).  
\item On the other hand, you do need to make explicit justification of the logical steps in a proof.  As stated above, you should cite a previous definition, theorem, etc. when necessary.
\item Similar to making it clear where your proof begins, you should indicate where it ends.  It is common to conclude a proof with the standard ``proof box" ($\square$ or $\blacksquare$).  This little square at end of a proof is sometimes called a \textbf{tombstone} or \textbf{Halmos symbol} after Hungarian-born American mathematician \href{https://en.wikipedia.org/wiki/Paul_Halmos}{Paul Halmos} (1916--2006).
\end{itemize}

It is of utmost importance that you work to understand every proof.  Questions---asked to your instructor, your peers, and yourself---are often your best tool for determining whether you understand a proof.  Another way to help you process and understand a proof is to try and make observations and connections between different ideas, proof statements and methods, and to compare various approaches. Observations might sound like some of the following:
\begin{itemize}
\item When I tried this proof, I thought I needed to \ldots But I didn't need that, because \ldots
\item I think that \ldots is important to this proof, because \ldots
\item When I read the statement of this theorem, it seemed similar to this earlier theorem. Now I see that it [is/isn't] because \ldots
\end{itemize}

If you would like additional guidance before you dig in, look over the guidelines in Appendix~\ref{appendix:elements_of_style}: Elements of Style for Proofs. It is suggested that you review this appendix occasionally as you progress through the book as some guidelines may not initially make sense or seem relevant. 

Now, go have fun and start exploring mathematics!

\epigraph{Our greatest glory is not in never falling, but in rising every time we fall.}{Confucius, philosopher}

\end{section}
\include*{IntroToMath}
\include*{TasteNumberTheory}
\include*{IntroToLogic}
\include*{NegatingAndContradiction}
\include*{IntroQuantification}
\include*{MoreQuantification}
\include*{SetTheory}
\include*{Sets}
\include*{RussellsParadox}
\include*{PowerSets}
\include*{IndexingSets}
\include*{CartesianProducts}
\include*{RealNumbers}
\include*{AxiomsRealNumbers}
\include*{Topology}
\include*{Induction}
\include*{IntroInduction}
\include*{MoreInduction}
\include*{CompleteInduction}
\include*{ThreeFamousTheorems}
\include*{FundamentalTheoremArithmetic}
\include*{IrrationalityRoot2}
\include*{InfinitudeOfPrimes}
\include*{RelationsPartitions}
\include*{Relations}
\include*{EquivalenceRelations}
\include*{Partitions}
\include*{ModularArithmetic}
%\include*{Posets}
\include*{Functions}
\include*{IntroFunctions}
\include*{InjectiveSurjectiveFunctions} 
\include*{CompositionsInverses}
\include*{ImagesInverseImages}
\include*{ContinuousFunctions}
\include*{Cardinality}
\include*{IntroCardinality}
\include*{FiniteSets}
\include*{InfiniteSets}
\include*{CountableSets}
\include*{UncountableSets}
%Appendices
\include{ElementsOfStyle}
\chapter{Fancy Mathematical Terms}
\label{appendix:fancy_math_terms}

Here are some important mathematical terms that you will encounter throughout mathematics.

\begin{enumerate}
\item \textbf{Definition}---a precise and unambiguous description of the meaning of a mathematical term.  It characterizes the meaning of a word by giving all the properties and only those properties that must be true.
\item \textbf{Theorem}---a mathematical statement that is proved using rigorous mathematical reasoning.  In a mathematical paper, the term theorem is often reserved for the most important results.
\item \textbf{Proposition}---a proved and often interesting result, but generally less important than a theorem.
\item \textbf{Lemma}---a minor result whose sole purpose is to help in proving a theorem.  It is a stepping stone on the path to proving a theorem. Occasionally lemmas can take on a life of their own (Zorn's Lemma, Urysohn's Lemma, Burnside's Lemma, Sperner's Lemma).
\item \textbf{Corollary}---a result in which the (usually short) proof relies heavily on a given theorem (we often say that ``this is a corollary of Theorem A'').
\item \textbf{Conjecture}---a statement that is unproved, but is believed to be true (Collatz Conjecture, Goldbach Conjecture, Twin prime Conjecture).
\item \textbf{Claim}---an assertion that is then proved.  It is often used like an informal lemma.
\item \textbf{Counterexample}---a specific example showing that a statement is false.
\item \textbf{Axiom/Postulate}---a statement that is assumed to be true without proof. These are the basic building blocks from which all theorems are proved (Euclid's five postulates, axioms of ZFC, Peano axioms).
\item \textbf{Identity}---a mathematical expression giving the equality of two (often variable) quantities (trigonometric identities, Euler's identity).
\item \textbf{Paradox}---a statement that can be shown, using a given set of axioms and definitions, to be both true and false. Paradoxes are often used to show the inconsistencies in a flawed axiomatic theory (e.g., Russell's Paradox).  The term paradox is also used informally to describe a surprising or counterintuitive result that follows from a given set of rules (Banach-Tarski Paradox, Alabama Paradox, Gabriel's Horn).
\end{enumerate}
\chapter{Paradoxes}
\label{appendix:paradoxes}

A \textbf{paradox} is a statement that can be shown, using a given set of axioms and definitions, to be both true and false. Recall that an axiom is a statement that is assumed to be true without proof. These are the basic building blocks from which all theorems are proved. Paradoxes are often used to show the inconsistencies in a flawed axiomatic theory.  The term paradox is also used informally to describe a surprising or counterintuitive result that follows from a given set of rules. In Section~\ref{sec:RussellsParadox}, we encountered two paradoxes:
\begin{itemize}
\item The Barber of Seville (Problem~\ref{prob:barber})
\item Russell's Paradox (Problem~\ref{prob:russell})
\end{itemize}
Below are several additional paradoxes that are worth exploring.

%The following paradoxes are from Dave Richeson.
\begin{enumerate}
\item \textbf{Librarian's Paradox.} A librarian is given the unenviable task of creating two new books for the library. Book A contains the names of all books in the library that reference themselves and Book B contains the names of all books in the library that do not reference themselves. But the librarian just created two new books for the library, so their titles must be in either Book A or Book B. Clearly Book A can be listed in Book B, but where should the librarian list Book B?

\item \textbf{Liar's Paradox.} Consider the statement: this sentence is false. Is it true or false?

\item \textbf{Berry Paradox.} Consider the claim: every natural number can be unambiguously described in fourteen words or less. It seems clear that this statement is false, but if that is so, then there is some smallest natural number which cannot be unambiguously described in fourteen words or less. Let's call it $n$. But now $n$ is ``the smallest natural number that cannot be unambiguously described in fourteen words or less.'' This is a complete and unambiguous description of $n$ in fourteen words, contradicting the fact that $n$ was supposed not to have such a description. Therefore, all natural numbers can be unambiguously described in fourteen words or less!

\item \textbf{The Naming Numbers Paradox.} Consider the claim: every natural number can be unambiguously described using no more than 50 characters (where a character is a--z, 0--9, and a ``space''). For example, we can describe 9 as ``9'' or ``nine'' or ``the square of the second prime number.'' There are only 37 characters, so we can describe at most $37^{50}$ numbers, which is very large, but not infinite. So the statement is false. However, here is a ``proof'' that it is true. Let $S$ be the set of natural numbers that can be unambiguously described using no more than 50 characters. For the sake of contradiction, suppose it is not all of $\mathbb{N}$. Then there is a smallest number $t\in\mathbb{N}\setminus S$. We can describe $t$ as: the smallest natural number not in $S$.  Thus $t$ can be described using no more than 50 characters. So $t\in S$, a contradiction.

\item \textbf{Euathlus and Protagoras.} Euathlus wanted to become a lawyer but could not pay Protagoras. Protagoras agreed to teach him under the condition that if Euathlus won his first case, he would pay Protagoras, otherwise not. Euathlus finished his course of study and did nothing. Protagoras sued for his fee. He argued:\\

\noindent If Euathlus loses this case, then he must pay (by the judgment of the court).\\
If Euathlus wins this case, then he must pay (by the terms of the contract).\\
He must either win or lose this case.\\
Therefore Euathlus must pay me.\\

\noindent But Euathlus had learned well the art of rhetoric. He responded:\\

\noindent If I win this case, I do not have to pay (by the judgment of the court).\\
If I lose this case, I do not have to pay (by the contract).\\
I must either win or lose the case.\\
Therefore, I do not have to pay Protagoras.
\end{enumerate}
\include{Definitions}

\end{document}
