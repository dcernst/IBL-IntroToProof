\begin{section}{Negating Implications and Proof by Contradiction}

So far we have discussed how to negate propositions of the form $A$, $A\wedge B$, and $A\vee B$ for propositions $A$ and $B$.  However, we have yet to discuss how to negate propositions of the form $A\implies B$.

\begin{problem}\label{prob:implication as disjunction}
Let $A$ and $B$ be propositions.  Conjecture an equivalent way of expressing the conditional proposition $A\implies B$ as a proposition involving the disjunction symbol $\vee$ and possibly the negation symbol $\neg$, but not the implication symbol $\implies$.  Prove your conjecture using a truth table.
\end{problem}

\begin{exercise}\label{exer:Darth Vader}
Let $A$ and $B$ be the propositions ``Darth Vader is a hippie" and ``Sarah Palin is a liberal", respectively.  Using Problem \ref{prob:implication as disjunction}, express $A\implies B$ as an English sentence involving the disjunction ``or."
\end{exercise}

\begin{problem}[*]\label{prob:negation of implication}
Let $A$ and $B$ be two propositions.  Conjecture an equivalent way of expressing the proposition $\neg(A\implies B)$ as a proposition involving the conjunction symbol $\wedge$ and possibly the negation symbol $\neg$, but not the implication symbol $\implies$.  Prove your conjecture using previous results.
\end{problem}

\begin{exercise}
Let $A$ and $B$ be the propositions in Exercise \ref{exer:Darth Vader}.  Using Problem \ref{prob:negation of implication}, express $\neg(A\implies B)$ as an English sentence involving the conjunction ``and."
\end{exercise}

\begin{exercise}
The following proposition is \emph{false}.  Negate this proposition to obtain a true statement.  Write your statement as a conjunction.
\begin{quote}
If $.\overline{99}=\frac{9}{10}+\frac{9}{100}+\frac{9}{1000}+\cdots$, then $.\overline{99}\neq 1$.
\end{quote}
You do \emph{not} need to prove your new statement.
\end{exercise}

Recall that a proposition is exclusively either true or false.  That is, a proposition can never be both true and false.  This idea leads us to the next definition.

\begin{definition}
A compound proposition that is always false is called a \textbf{contradiction}.  A compound statement that is always true is called a \textbf{tautology}.
\end{definition}

\begin{theorem}
Let $A$ be a proposition. Then $\neg A\wedge A$ is a contradiction.
\end{theorem}

\begin{exercise}
Provide an example of a tautology.
\end{exercise}

%student feedback is that this part needs to be improves.  Perhaps add a skeleton proof?
Suppose that we want to prove some proposition $P$ (which might be something like $A\implies B$ or possibly more complicated).  One approach, called \textbf{proof by contradiction}, involves assuming $\neg P$ and then logically deducing a contradiction of the form $Q\wedge \neg Q$, where $Q$ is some proposition (possibly equal to $P$).  Since this is absurd, it cannot be the case that $\neg P$ is true, which implies that $P$ is true.

Among other situations, proof by contradiction can be useful for proving statements of the form $A\implies B$, where $B$ is worded negatively or $\neg B$ is easier to ``get your hands on."

\begin{question}
Let $A$ and $B$ be propositions.  Describe a general strategy for proving $A\implies B$ via proof by contradiction.
\end{question}

Prove the following theorem in two ways: (i) prove the contrapositive and (ii) use proof by contradiction.

\begin{theorem}[*]
Assume that $x\in\mathbb{Z}$.  If $x$ is odd, then 2 does not divide $x$. (Prove in two different ways.)
\end{theorem}

Prove the following theorem by contradiction.

\begin{theorem}[*]
Assume that $x,y\in\mathbb{N}$.  If $x$ divides $y$, then $x\leq y$. (Prove using a proof by contradiction.)
\end{theorem}

\begin{question}
What obstacles (if any) are there to proving the previous theorem directly without using proof by contradiction?
\end{question}

\end{section}
