\documentclass[11pt]{article}%{amsart}

\usepackage[margin=1in]{geometry}
\usepackage{url}
\usepackage[breaklinks]{hyperref}
\usepackage{color}
\hypersetup{
	colorlinks=true,
	linkcolor=darkblue,
	anchorcolor=darkblue,
	citecolor=darkblue,
	pagecolor=darkblue,
	urlcolor=darkblue,
	pdftitle={},
	pdfauthor={},
    bookmarksnumbered
}
\definecolor{darkblue}{rgb}{0, 0, .6}
\usepackage{tikz}
\usepackage[framemethod=TikZ]{mdframed}

\mdfdefinestyle{skeleton}{roundcorner=5pt,leftmargin=5pt,rightmargin=5pt,backgroundcolor = black!5!white}

\setlength{\parindent}{0pt}
\setlength{\fboxsep}{10pt}

\newcommand{\blankline}{\pagebreak[2]\vspace{.5\baselineskip}}

%%%%%%%%%%%%%%%%%%%

\begin{document}

\title{An Introduction to Proof via Inquiry-Based Learning\\
Instructor Guide}
\author{Dana C.~Ernst}
\date{\today}

\maketitle

\begin{mdframed}[style=skeleton]
This Instructor Guide is a work in progress!  If you have questions about something not addressed in this guide, you can either submit an issue on \href{https://github.com/dcernst/IBL-IntroToProof/issues}{GitHub} or send me an email at \url{dana.ernst@nau.edu}.  I want instructors that utilize \emph{An Introduction to Proof via Inquiry-Based Learning} to be as successful as possible, so please reach out with questions or concerns.
\end{mdframed}

\section*{Overview}

\emph{An Introduction to Proof via Inquiry-Based Learning} is intended to be used for a one-semester or quarter introduction to proof course (sometimes referred to as a transition to proof course). The purpose of the book is to introduce the reader to the process of constructing and writing formal and rigorous mathematical proofs. The intended audience is mathematics majors and minors. However, the book is also appropriate for anyone curious about mathematics and writing proofs. Most users of the book will have taken at least one semester of calculus, although other than some familiarity with a few standard functions in Chapter~8: Functions, content knowledge of calculus is not required. 

\blankline

In order to promote a more active participation in student learning, \emph{An Introduction to Proof via Inquiry-Based Learning} adheres to an educational philosophy called inquiry-based learning (IBL). IBL is a student-centered method of teaching that engages students in sense-making activities and challenges them to create or discover mathematics.  The book expects readers to actively engage with the topics at hand and to construct their own understanding.  The reader will be given tasks requiring them to solve problems, conjecture, experiment, explore, create, and communicate.  Rather than showing facts or a clear, smooth path to a solution, the book guides and mentors the reader through an adventure in mathematical discovery. 

\blankline

However, the book makes no assumptions about the specifics of how the instructor will choose to implement an IBL approach. Ultimately, the instructor should do what is best for their students. Generally speaking, students are told which problems and theorems to grapple with for the next class sessions, and then the majority of class time is devoted to students working in groups on unresolved solutions/proofs or having students present their proposed solutions/proofs to the rest of the class. Students should---as much as possible---be responsible for guiding the acquisition of knowledge and validating the ideas presented. That is, you should not be looking to the instructor as the sole authority. In an IBL course, instructor and students have joint responsibility for the depth and progress of the course. While effective IBL courses come in a variety of forms, they all possess a few essential ingredients. According to \href{https://www.colorado.edu/eer/sites/default/files/attached-files/laursenrasmussencommentaryauthorversion0219.pdf}{Laursen and Rasmussen (2019)}, the Four Pillars of IBL are:
\begin{itemize}
\item Students engage deeply with coherent and meaningful mathematical tasks.
\item Students collaboratively process mathematical ideas.
\item Instructors inquire into student thinking.
\item Instructors foster equity in their design and facilitation choices.
\end{itemize}
The book can only address the first pillar while it is the responsibility of the instructor and class to develop a culture that provides an adequate environment for the remaining pillars to take root. Again, I would like to emphasize that the book is mostly agnostic about the approach that an instructor would take when teaching out of the book.  Heck, I don't see any reason why an instructor couldn't use the book for a lecture-based class. Although that's certainly not what I had in mind when writing the book.

\blankline

The book includes more content than one can expect to cover in a single semester or quarter. This allows the instructor/reader to pick and choose the sections that suit their needs and desires. Each chapter takes a focused approach to the included topics, but also includes many gentle exercises aimed at developing intuition.

\blankline

The following sections form the core of the book and are likely the sections that an instructor would focus on in a one-semester introduction to proof course.
\begin{itemize}
\item Chapter~2: Mathematics and Logic. All sections.
\item Chapter~3: Set Theory. Sections 3.1, 3.3, 3.4, and 3.5.
\item Chapter~4: Induction. All sections.
\item Chapter~7: Relations and Partitions. Sections 7.1, 7.2, and 7.3.
\item Chapter~8: Functions. Sections 8.1, 8.2, 8.3, and 8.4.
\item Chapter~9: Cardinality. All sections.
\end{itemize}
Time permitting, instructors can pick and choose topics from the remaining sections.  I typically cover the core sections listed above together with Chapter~6: Three Famous Theorems during a single semester. 

\blankline

There are many useful resources available that instructors can utilize for designing an effective IBL/active learning experience for their students.  The \href{http://www.inquirybasedlearning.org}{Academy of Inquiry Based Learning} is a good place to get started.  I also suggest consulting the MAA's \href{https://www.maa.org/programs-and-communities/curriculum%20resources/instructional-practices-guide}{Instructional Practices Guide}, which is a guide to evidence-based instructional practices in undergraduate mathematics.  One effective approach to getting started with IBL is mimicking another instructor's approach and then refining for your purposes over time.  Feel free to borrow as many ideas as you would like from how I set up the course I teach using \emph{An Introduction to Proof via Inquiry-Based Learning}.  You can find the syllabus, homework assignments, etc from two recent iterations of my course at the following links:
\begin{itemize}
\item \href{http://danaernst.com/teaching/mat320f21/}{Fall 2021}
\item \href{http://danaernst.com/teaching/mat320s20/}{Spring 2020}
\end{itemize}
My Fall 2021 course utilized a version of the book that is nearly identical to the current version, but I happened to cover far less material than usual.  I covered more ground in Spring 2020, but the version of the book I used that semester may look a bit different than the current version.  I would consider the amount of material I covered in Spring 2020 to be fairly typical.  The differences in the amount of material that I covered had nothing to do with the version of the book!  Feel free to reach out with questions about how to set up your course or how to best make use of the book.

\section*{Structure of the Textbook}

As students read the book, they should be digesting the material in a meaningful way.  In addition to reading and understanding new definitions and their related concepts, students will be asked to complete problems aimed at solidifying their understanding of the material.  In particular, the reader is asked to make conjectures, produce counterexamples, and prove theorems. All of these tasks will almost always be challenging.

\blankline

The items labeled as \textbf{Definition} and \textbf{Example} are meant to be read and digested.  However, the items labeled as \textbf{Problem}, \textbf{Theorem}, and \textbf{Corollary} require action on the reader's part.  Items labeled as \textbf{Problem} are sort of a mixed bag. Some Problems are computational in nature and aimed at improving understanding of a particular concept while others ask the reader to provide a counterexample for a statement if it is false or to provide a proof if the statement is true. Items with the \textbf{Theorem} and \textbf{Corollary} designation are mathematical facts and the intention is for the reader to produce a valid proof of the given statement. All of this is spelled out in the Introduction of the book, but I suggest taking the time to communicate this to your students.

\blankline

Oftentimes, the problems and theorems are guiding the reader towards a substantial, more  general result. Other times, they are designed to get the reader to apply ideas in a new way. Please take the time to tell your students that every task in the book is doable but sometimes very challenging.  They may not be successful on their first, or even second or third try.  This is okay!  Remind them of this often. 

\blankline

Discussion of new topics is typically kept at a minimum and there are very few examples in the book. This is intentional.  One of the objectives of the items labeled as \textbf{Problem} is for the reader to produce the examples needed to internalize unfamiliar concepts.  The overarching goal of the book is to help the reader develop a deep and meaningful understanding of the processes of producing mathematics by putting them in direct contact with mathematical phenomena.

\section*{Evidence in Favor of Active Learning}

If you are already using or considering using \emph{An Introduction to Proof via Inquiry-Based Learning}, you likely don't need to be convinced of the merits of active learning.  Nonetheless, below is very brief summary of some the data that supports the use of active learning.

\blankline

Evidence in favor of some form of active engagement of students is strong across STEM disciplines. \href{https://pubmed.ncbi.nlm.nih.gov/24821756/}{Freeman et al.~(2014)} conducted a meta-analysis of 225 studies of various forms of active learning, and found that students were 1.5 times more likely to fail in traditional courses as compared to active learning courses, and students in active learning courses outperformed students in traditional courses by 0.47 standard deviations on examinations and concept inventories. The following snippet from Freeman et al.~(2014) captures the importance of utilizing active learning across STEM education:
\begin{quote}
\emph{``The results raise questions about the continued use of traditional lecturing as a control in research studies, and support active learning as the preferred, empirically validated teaching practice in regular classrooms."}
\end{quote}

For IBL specifically, a research group from the University of Colorado Boulder led by Sandra Laursen conducted a comprehensive study of student outcomes in IBL undergraduate mathematics courses while linking these outcomes to students' and instructors' experiences of IBL (see \href{https://www.colorado.edu/eer/research-areas/student-centered-stem-education/inquiry-based-learning-college-mathematics}{Laursen et al.~2011; Laursen 2013; Kogan and Laursen 2014; Laursen et al.~2014}). This quasi-experimental, longitudinal study examined over 100 courses at four different campuses over a period that spanned two years.

\blankline

On average over 60\% of IBL class time was spent on student-centered activities including student-led presentations, discussion, and small-group work. In contrast, in non-IBL courses, 87\% of class time was devoted to students' listening to an instructor talk. In addition, the IBL sections were rated more highly for a supportive classroom environment and students conveyed that engaging in meaningful mathematical tasks while collaborating was essential to their learning. Below is a brief summary of some of the outcomes of Laursen et al.'s work.
\begin{itemize}
\item After an IBL or comparative course, IBL students reported higher learning gains than their non-IBL peers, across cognitive, affective, and collaborative domains of learning.
\item In later courses, students who had taken an IBL course earned grades as good or better than those of students who took non-IBL sections, despite having ``covered" less material.
\item Non-IBL courses show a marked gender gap: across the board, women reported lower learning gains and less supportive attitudes than did men (effect size 0.4--0.5). Women's confidence and sense of mastery of mathematics, and their interest in continued study of math were lower. This difference appears to be primarily affective, not due to real differences in women's mathematical preparation or achievement.
\item This gender gap was erased in IBL classes: women's learning gains were equal to men's, and their confidence and intent to persist similar. IBL approaches leveled the playing field for women, fixing a course that is problematic for women yet with no harm to men.
\end{itemize}

You can watch a short YouTube video of Sandra Laursen summarizing most of the recent research about inquiry-based learning \href{https://www.youtube.com/watch?v=m_HK6b3RGOc&feature=youtu.be}{here}. The \href{https://www.cbmsweb.org}{Conference Board of the Mathematical Sciences (CBMS)} wrote the following in their \href{https://www.cbmsweb.org/2016/07/active-learning-in-post-secondary-mathematics-education/}{position statement on active learning in 2016}:

\begin{quote}
\emph{``\ldots we call on institutions of higher education, mathematics departments and the mathematics faculty, public policy-makers, and funding agencies to invest time and resources to ensure that effective active learning is incorporated into post-secondary mathematics classrooms."}
\end{quote}

In addition, the Manifesto of the \href{https://www.maa.org/programs-and-communities/curriculum%20resources/instructional-practices-guide}{MAA Instructional Practices Guide} states: 

\begin{quote}
\emph{``We must gather the courage to advocate beyond our own classroom for student-centered instructional strategies that promote equitable access to mathematics for all students. We stand at a crossroads, and we must choose the path of transformation in order to fulfill our professional responsibility to our students."}
\end{quote}

\section*{Chapter 1: Introduction}

I ask my students to read this short chapter and then take the time in class to highlight the content in Section~1.4: Structure of the Textbook.  Despite this, I always have at least one student that asks me what they are supposed to do the first time I assign a theorem.  At a minimum, I would ask students to read Sections~1.2--1.4.

\section*{Chapter 2: Mathematics and Logic}

\section*{Chapter 3: Set Theory}

\section*{Chapter 4: Induction}

\section*{Chapter 5: The Real Numbers}

\section*{Chapter 6: Three Famous Theorems}

\section*{Chapter 7: Relations and Partitions}

\section*{Chapter 8: Functions}

\section*{Chapter 9: Cardinality}

\end{document}