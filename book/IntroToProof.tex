\documentclass[letterpaper,11pt,onesided]{book}%

\newif\ifinstructor
%% uncomment one of the following, depending if you want to include the
%% instructor notes or not
\instructortrue
%\instructorfalse


\newif\ifnotes
%% uncomment one of the following, depending if you want to include the
%% instructor notes or not
\notestrue
%\notesfalse

\usepackage{tabls}
\usepackage{booktabs}
\usepackage{amsmath}
\usepackage{amssymb}
\usepackage{amsthm}
\usepackage{amsfonts}
\usepackage{multicol}
\usepackage{enumitem}
\usepackage{stmaryrd}
\usepackage{wasysym}
\usepackage{microtype}
\usepackage{tikz}
\usetikzlibrary{positioning}
\usepackage[hang,flushmargin,symbol*]{footmisc}
\usepackage{comment}
\usepackage{graphicx}
\usepackage{wrapfig}
\usepackage{color}
\definecolor{darkblue}{rgb}{0, 0, .6}
\definecolor{grey}{rgb}{.7, .7, .7}
\usepackage[breaklinks]{hyperref}
\hypersetup{
	colorlinks=true,
	linkcolor=darkblue,
	anchorcolor=darkblue,
	citecolor=darkblue,
	pagecolor=darkblue,
	urlcolor=darkblue,
	pdftitle={},
	pdfauthor={}
}

\let\oldsection\section
\renewcommand\section{\clearpage\oldsection}


\usepackage[margin=1in]{geometry}

%%%%%%%%%%%%%%%%%%%%%%%%%%%%%%%%%%%%%%%%%%%%%%%%%%%%%%%%%%%%%%%%%%%%%%%%%%%%%%%%%%%%%%%%%%%%%%%%%%%%%%
% Margin notes
%%%%%%%%%%%%%%%%%%%%%%%%%%%%%%%%%%%%%%%%%%%%%%%%%%%%%%%%%%%%%%%%%%%%%%%%%%%%%%%%%%%%%%%%%%%%%%%%%%%%%%
%\usepackage[left=1in,right=2.75in,top=1in,bottom=1in]{geometry}
%\marginparwidth 1.75in
%\let\oldmarginpar\marginpar
%\renewcommand\marginpar[1]{\-\oldmarginpar{\raggedright\footnotesize #1}}
%\renewcommand\marginpar[1]{\-\oldmarginpar[\raggedleft\footnotesize #1]{\raggedright\footnotesize #1}}




%%%%%%%%%%%%%%%%%%%%%%%%%%%%%%%%%%%%%%%%%%%%%%%%%%%%%%%%%%%%%%%%%%%%%%%%%%%%%%%%%%%%%%%%%%%%%%%%%%%%%%
% Embed the date and time into every page
%%%%%%%%%%%%%%%%%%%%%%%%%%%%%%%%%%%%%%%%%%%%%%%%%%%%%%%%%%%%%%%%%%%%%%%%%%%%%%%%%%%%%%%%%%%%%%%%%%%%%%
%\usepackage[12hr]{datetime}
%\newdateformat{draftdate}{%
%\shortdayofweekname{\THEDAY}{\THEMONTH}{\THEYEAR}, %
%\THEDAY\ \shortmonthname[\THEMONTH] \THEYEAR}
%\draftdate
%\usepackage{eso-pic}
%\AddToShipoutPicture{\put(10,10){\small Draft: \today\ at \currenttime }}%--- version: \MakeUppercase{\svnInfoRevision}}}

%%%%%%%%%%%%%%%%%%%%%%%%%%%%%%%%%%%%%%%%%%%%%%%%%%%%%%%%%%%%%%%%%%%%%%%%%%%%%%%%%%%%%%%%%%%%%%%%%%%%%%
% Instructor-specific material (answers, helps, etc.)
%%%%%%%%%%%%%%%%%%%%%%%%%%%%%%%%%%%%%%%%%%%%%%%%%%%%%%%%%%%%%%%%%%%%%%%%%%%%%%%%%%%%%%%%%%%%%%%%%%%%%%
\ifinstructor
  \newcommand{\instructor}[1]{\marginpar{\textbf{Instructor: }#1}}
\else
  \newcommand{\instructor}[1]{}
\fi

%%%%%%%%%%%%%%%%%%%%%%%%%%%%%%%%%%%%%%%%%%%%%%%%%%%%%%%%%%%%%%%%%%%%%%%%%%%%%%%%%%%%%%%%%%%%%%%%%%%%%%
% Personal notes about improving the text, etc.
%%%%%%%%%%%%%%%%%%%%%%%%%%%%%%%%%%%%%%%%%%%%%%%%%%%%%%%%%%%%%%%%%%%%%%%%%%%%%%%%%%%%%%%%%%%%%%%%%%%%%%
\ifnotes
\renewcommand{\thefootnote}{\roman{footnote}}
\newcommand{\note}[1]{\footnote{#1}\marginpar{\fbox{\textbf{\thefootnote}}}}
\else
\newcommand{\note}[1]{}
\fi



%%%%%%%%%%%%%%%%%%%%%%%%%%%%%%%%%%%%%%%%%%%%%%%%%%%%%%%%%%%%%%%%%%%%%%%%%%%%%%%%%%%%%%%%%%%%%%%%%%%%%%
% Commands to define a term (so that we get nice indexing
%%%%%%%%%%%%%%%%%%%%%%%%%%%%%%%%%%%%%%%%%%%%%%%%%%%%%%%%%%%%%%%%%%%%%%%%%%%%%%%%%%%%%%%%%%%%%%%%%%%%%%
\newcommand{\define}[2][]{\textbf{#2}\index{#1#2}}
\newcommand{\definemargin}[2][]{\marginpar{#2}\define[#1]{#2}}

%%%%%%%%%%%%%%%%%%%%%%%%%%%%%%%%%%%%%%%%%%%%%%%%%%%%%%%%%%%%%%%%%%%%%%%%%%%%%%%%%%%%%%%%%%%%%%%%%%%%%%
% Theorems
%%%%%%%%%%%%%%%%%%%%%%%%%%%%%%%%%%%%%%%%%%%%%%%%%%%%%%%%%%%%%%%%%%%%%%%%%%%%%%%%%%%%%%%%%%%%%%%%%%%%%%

\newtheoremstyle{box}%
{}{}% standard spacing before and after
{}% Body style
{}{\bfseries}{.}% Heading indent, font, and punctuation
{ }% space after heading
{\thmname{#1}\thmnumber{ #2}\thmnote{: #3}}% head spec

\newtheoremstyle{problem}%
{}{}% standard spacing before and after
{}% Body style
{}{\bfseries}{}% Heading indent, font, and punctuation
{1em}% space after heading
{\fbox{\thmname{#1}\thmnumber{ #2}}\thmnote{ (#3)}}% head spec


\theoremstyle{problem}
\newtheorem{theorem}{Theorem}[chapter]
\newtheorem{question}[theorem]{Question}
\newtheorem{problem}[theorem]{Problem}
\newtheorem{exercise}[theorem]{Exercise}
\newtheorem{corollary}[theorem]{Corollary}

\theoremstyle{definition}
\newtheorem{acknowledgement}[theorem]{Acknowledgement}
\newtheorem{algorithm}[theorem]{Algorithm}
\newtheorem{axiom}[theorem]{Axiom}
\newtheorem{case}[theorem]{Case}
\newtheorem{claim}[theorem]{Claim}
\newtheorem{conclusion}[theorem]{Conclusion}
\newtheorem{condition}[theorem]{Condition}
\newtheorem{conjecture}[theorem]{Conjecture}
\newtheorem{criterion}[theorem]{Criterion}
\newtheorem{definition}[theorem]{Definition}
\newtheorem{example}[theorem]{Example}
\newtheorem{journal}[theorem]{Journal}
\newtheorem{lemma}[theorem]{Lemma}
\newtheorem{notation}[theorem]{Notation}
\newtheorem{proposition}[theorem]{Proposition}
\newtheorem{remark}[theorem]{Remark}
\newtheorem{solution}[theorem]{Solution}
\newtheorem{summary}[theorem]{Summary}
\newtheorem{skeleton}[theorem]{Skeleton Proof}


\newsavebox{\savepar}
\newenvironment{textbox}{\noindent\begin{lrbox}{\savepar}\begin{minipage}[c]{.98\textwidth}}{\end{minipage}\end{lrbox}\fcolorbox{black}{white}{\usebox{\savepar}}}


%%%%%%%%%%%%%%%%%%%%%%%%%%%%%%%%%%%%%%%%%%%%%%%%%%%%%%%%%%%%%%%%%%%%%%%%%%%%%%%%%%%%%%%%%%%%%%%%%%%%%%
% Notation shortcuts
%%%%%%%%%%%%%%%%%%%%%%%%%%%%%%%%%%%%%%%%%%%%%%%%%%%%%%%%%%%%%%%%%%%%%%%%%%%%%%%%%%%%%%%%%%%%%%%%%%%%%%

\DeclareMathOperator{\dom}{Dom}
\DeclareMathOperator{\codom}{Codom}
\DeclareMathOperator{\range}{Rng}
\newcommand{\blank}[1]{\raisebox{0pt}[14pt]{\rule{#1}{0.7pt}}}

\let\origdoublepage\cleardoublepage

\newcommand{\clearemptydoublepage}{%
  \clearpage
  {\pagestyle{empty}\origdoublepage}%
}



\begin{document}
\frontmatter
\title{Course Notes for Introduction to Proof}
\author{D.C.~Ernst\thanks{Northern Arizona University, \url{Dana.Ernst@nau.edu}}}
\date{Typeset on \today\\
\vfill
  Part of this work is an adaptation of notes written by Stan Yoshinobu of Cal Poly and Matthew Jones of California State University, Dominguez Hills.
\vfill
\includegraphics[height=1.3cm]{by-sa}
\vfill}
\maketitle
 \thispagestyle{empty}

% \noindent\copyright{ year author.  Some Rights Reserved.\\

% \bigskip

% \noindent Except where otherwise noted, this work is licensed under the Creative Commons
% Attribution-ShareAlike 3.0 United States License. To view a copy of
% this license, visit 
% \begin{center}
%   \url{http://creativecommons.org/licenses/by-sa/3.0/us/}
% \end{center}
% or send a letter to Creative Commons, 171 Second Street, Suite 300,
% San Francisco, California, 94105, USA.

% \bigskip

% \noindent Please attribute this work to:
% \smallskip

%  Dana Ernst (Northern Arizona University),  Stan Yoshinobu (Cal Poly), and Matthew Jones (California State University, Dominguez Hills)

% \vfill 

% The source for this work is available at \url{https://github.com/jasongrout/IBL-IntroToProof}.
% \vfill
% }
\tableofcontents
\clearemptydoublepage

\begin{chapter}{Preface}
This is preface text.
\end{chapter}

\clearemptydoublepage

\mainmatter

\begin{chapter}{Introduction to Mathematics}
\begin{section}{A Taste of Number Theory}

%This section (or possibly a new section before this one) really needs more of a lead in.  For example, I should add:
%What is a theorem, proof, axiom, definition?
%What are the ground rules for doing presentations?
%Description of a direct proof.
%Discuss of implicit conditionals.

In this section, we will work with the set of integers, $\mathbb{Z} = \{\ldots, -3, -2, -1, 0, 1, 2, 3, \ldots\}$.  The purpose of this section is to get started with proving some theorems about numbers and study the properties of $\mathbb{Z}$.

It is important to note that we are diving in head first here.  There are going to be some subtle issues that you will bump into and our goal will be to see what those issues are, and then we will take a step back and start again.  See what you can do!

\begin{definition}
An integer $n$ is \textbf{even} if $n=2k$ for some integer $k$. \end{definition}

\begin{definition}
An integer $n$ is \textbf{odd} if $n=2k+1$ for some integer $k$. \end{definition}

\begin{theorem}\label{two consecutive ints}
The sum of two consecutive integers is odd.
\end{theorem}

\begin{theorem}
If $n$ is even, then $n^2$ is even.
\end{theorem}

\begin{problem}[*]\label{sum of even and odd}
Prove or provide a counterexample:  The sum of an even number and an odd number is odd.
\end{problem}

\begin{question}
Did Theorem~\ref{two consecutive ints} need to come before Problem~\ref{sum of even and odd}?  Could we have used Problem~\ref{sum of even and odd} to prove Theorem~\ref{two consecutive ints}?  If so, outline how this alternate proof would go.  Perhaps your original proof utilized the approach I'm hinting at.  If this is true, can you think of a proof that does not rely directly on Problem~\ref{sum of even and odd}?  Is one approach better than the other?
\end{question}

\begin{problem}
Prove or provide a counterexample: The product of an odd number and an even number is odd.
\end{problem}

\begin{problem}[*]
Prove or provide a counterexample: The product of an odd number and an odd number is odd.
\end{problem}

\begin{problem}[*]
Prove or provide a counterexample: The product of two even numbers is even.
\end{problem}

\begin{definition}
An integer $n$ \textbf{divides} the integer $m$, written $n|m$, if and only if there exists an integer $k$ such that $m=nk$. In the same context, we may also write that $m$ \textbf{is divisible by} $n$.
\end{definition}

In this section on number theory, we allow addition, subtraction, and multiplication.  Division is not allowed since an integer divided by an integer may result in a number that is not an integer. The upshot: don't write $\frac{m}{n}$.  When you feel the urge to divide, switch to an equivalent formulation using multiplication.

\begin{problem}
Let $n$ be an integer.  Prove or provide a counterexample: If 6 divides $n$, then 3 divides $n$.\end{problem}

\begin{problem}
Let $n$ be an integer.  Prove or provide a counterexample: If 6 divides $n$, then 4 divides $n$.
\end{problem}

\begin{theorem}[*]
Assume $n$, $m$, and $a$ are integers.  If $a|n$, then $a|mn$.
\end{theorem}

A theorem that follows almost immediately from another theorem is called a \textbf{corollary}.  See if you can prove the next result quickly using the previous theorem.  Be sure to cite the theorem in your proof.

\begin{corollary}
Assume $n$ and $a$ are integers.  If $a$ divides $n$, then $a$ divides $n^2$.
\end{corollary}

\begin{problem}
Assume $n$ and $a$ are integers.  Prove or provide a counterexample:  If $a$ divides $n^2$, then $a$ divides $n$.
\end{problem}

\begin{theorem}
Assume $a$ and $n$ are integers. If $a$ divides $n$, then $a$ divides $-n$. 
\end{theorem}

\begin{theorem}[*]
Assume $a$, $m$, and $n$ are integers. If $a$ divides $m$ and $a$ divides $n$, then $a$ divides $m+n$. 
\end{theorem}

Once we've proved a few theorems, we should be on the look out to see if we can utilize any of our current results to prove new results.  There's no point in reinventing the wheel if we don't have to.  Try to use a couple of our previous results to prove the next theorem.

\begin{theorem}
Assume $a$, $m$, and $n$ are integers. If $a$ divides $m$ and $a$ divides $n$, then $a$ divides $m-n$.
\end{theorem}

\begin{problem}[*]
Assume $a$, $b$, and $m$ are integers. Determine whether the following statement holds sometimes, always, or never.  If $ab$ divides $m$, then $a$ divides $m$ and $b$ divides $m$.  Justify with a proof or counterexample.
\end{problem}

\begin{theorem}[*]
If $a, b$, and $c$ are integers where $a$ divides $b$ and $b$ divides $c$, then $a$ divides $c$.
\end{theorem}

The previous theorem is referred to as \textbf{transitivity of division of integers}.

\begin{theorem}[*]
The sum of any three consecutive integers is always divisible by three.\end{theorem}

\end{section}

% unfortunately, this semester, there was an inconsistency in how
% theorems were numbered at the start of section 1.2.  Since we
% already passed out sheets for these, I am resetting the numbers
% here.
\addtocounter{theorem}{-1}
\input{1.2-intro-logic}
\begin{section}{Negating Implications and Proof by Contradiction}

So far we have discussed how to negate propositions of the form $A$, $A\wedge B$, and $A\vee B$ for propositions $A$ and $B$.  However, we have yet to discuss how to negate propositions of the form $A\implies B$.

\begin{problem}\label{prob:implication as disjunction}
Let $A$ and $B$ be propositions.  Conjecture an equivalent way of expressing the conditional proposition $A\implies B$ as a proposition involving the disjunction symbol $\vee$ and possibly the negation symbol $\neg$, but not the implication symbol $\implies$.  Prove your conjecture using a truth table.
\end{problem}

\begin{exercise}\label{exer:Darth Vader}
Let $A$ and $B$ be the propositions ``Darth Vader is a hippie" and ``Sarah Palin is a liberal", respectively.  Using Problem \ref{prob:implication as disjunction}, express $A\implies B$ as an English sentence involving the disjunction ``or."
\end{exercise}

\begin{problem}[*]\label{prob:negation of implication}
Let $A$ and $B$ be two propositions.  Conjecture an equivalent way of expressing the proposition $\neg(A\implies B)$ as a proposition involving the conjunction symbol $\wedge$ and possibly the negation symbol $\neg$, but not the implication symbol $\implies$.  Prove your conjecture using previous results.
\end{problem}

\begin{exercise}
Let $A$ and $B$ be the propositions in Exercise \ref{exer:Darth Vader}.  Using Problem \ref{prob:negation of implication}, express $\neg(A\implies B)$ as an English sentence involving the conjunction ``and."
\end{exercise}

\begin{exercise}
The following proposition is \emph{false}.  Negate this proposition to obtain a true statement.  Write your statement as a conjunction.
\begin{quote}
If $.\overline{99}=\frac{9}{10}+\frac{9}{100}+\frac{9}{1000}+\cdots$, then $.\overline{99}\neq 1$.
\end{quote}
You do \emph{not} need to prove your new statement.
\end{exercise}

Recall that a proposition is exclusively either true or false.  That is, a proposition can never be both true and false.  This idea leads us to the next definition.

\begin{definition}
A compound proposition that is always false is called a \textbf{contradiction}.  A compound statement that is always true is called a \textbf{tautology}.
\end{definition}

\begin{theorem}
Let $A$ be a proposition. Then $\neg A\wedge A$ is a contradiction.
\end{theorem}

\begin{exercise}
Provide an example of a tautology using arbitrary positions and any of the logical connectives $\neg$, $\wedge$, and $\vee$.  Then prove that your example is in fact a tautology.
\end{exercise}

Suppose that we want to prove some proposition $P$ (which might be something like $A\implies B$ or possibly more complicated).  One approach, called \textbf{proof by contradiction}, involves assuming $\neg P$ and then logically deducing a contradiction of the form $Q\wedge \neg Q$, where $Q$ is some proposition (possibly equal to $P$).  Since this is absurd, it cannot be the case that $\neg P$ is true, which implies that $P$ is true.  The tricky part about a proof by contradiction is that it is not usually obvious what the statement $Q$ is going to be.  Here is what the general structure for a proof by contradiction looks like.

\bigskip

\begin{skeleton}[Proof of $P$ by contradiction]
Here is what the general structure for a proof by contradiction looks like if we are trying to prove the proposition $P$.

\bigskip

\begin{textbox}
\begin{proof}
For sake of a contradiction, assume $\neg P$.
\begin{center}
$\vdots$\\
(Use definitions and previous theorems to derive some $Q$ and its negation $\neg Q$.)\\
$\vdots$
\end{center}
\noindent This is a contradiction.  Therefore, $P$.
\end{proof}
\end{textbox}

\end{skeleton}

Among other situations, proof by contradiction can be useful for proving statements of the form $A\implies B$, where $B$ is worded negatively or $\neg B$ is easier to ``get your hands on."  

\begin{skeleton}[Proof of $A\implies B$ by contradiction]\label{pf by contradiction for implication}
If you want to prove the proposition $A\implies B$ via a proof by contradiction, then the structure of the proof is as follows.

\bigskip

\begin{textbox}
\begin{proof}
For sake of a contradiction, assume $A$ and $\neg B$.
\begin{center}
$\vdots$\\
(Use definitions and previous theorems to derive some $Q$ and its negation $\neg Q$.)\\
$\vdots$
\end{center}
\noindent This is a contradiction.  Therefore, if $A$, then $B$.
\end{proof}
\end{textbox}
\end{skeleton}

\begin{question}
In Skeleton Proof \ref{pf by contradiction for implication}, why did we start by assuming $A$ and $\neg B$?
\end{question}

Prove the following theorem in two ways: (i) prove the contrapositive, and (ii) prove using a proof by contradiction.

\begin{theorem}[*]
Assume that $x\in\mathbb{Z}$.  If $x$ is odd, then 2 does not divide $x$. (Prove in two different ways.)
\end{theorem}

Prove the following theorem by contradiction.

\begin{theorem}[*]
Assume that $x,y\in\mathbb{N}$\footnote{Note that $\mathbb{N}=\{1,2,3,\ldots\}$ is the set of \textbf{natural numbers}. Notice that we did not include $0$ in the set of natural numbers.  It is worth pointing out that there is some disagreement about this---some mathematicians (like set theorists) include $0$ in $\mathbb{N}$, but this will not be our convention.  The given statement is not true if we replace $\mathbb{N}$ with $\mathbb{Z}$.  Do you see why?}.  If $x$ divides $y$, then $x\leq y$. (Prove using a proof by contradiction.)
\end{theorem}

\begin{question}
What obstacles (if any) are there to proving the previous theorem directly without using proof by contradiction?
\end{question}

\end{section}

% here, we skip 1 to get back on track (see above)...
\addtocounter{theorem}{1}
\input{1.4-quantification}
\begin{section}{More on Quantification}

In the last section, we introduced the universal quantifier ``for all'' and the existential quantifier ``there exists\ldots such that.''  Here are a couple of important points to remember about quantification:
\begin{enumerate}
\item In order to have a proposition, all variables must be bound.  That is, all variables must be quantified.  This can happen in at least two ways:
\begin{enumerate}
\item The variables are explicitly bound by quantifiers in the same sentence, or
\item The variables are implicitly bound by preceding sentences and/or by context.  \emph{Note:}  Statements of the form ``Let $x=\ldots$" and ``Let $x\in\ldots$" bind the variable $x$ and remove ambiguity.
\end{enumerate}
\item The order of the quantification is important.  Reversing the order of the quantifiers can substantially change the meaning of a proposition.
\end{enumerate}

Using our logical connectives (``and", ``or", ``If\ldots, then\ldots", and ``not") together with quantification, we can build very complex mathematical statements.

\begin{example}\label{ex:def limit}
Let $f$ be a function and consider the formal definition of the
calculus statement $\displaystyle\lim_{x\to c}f(x)=L$.
This statement about the limit of $f(x)$ at $x=c$ is equivalent to:
\begin{quote}
For all $\epsilon >0$, there exists $\delta >0$ such that for all $x$,
if $0<|x-c|<\delta$, then $|f(x)-L|<\epsilon$.
\end{quote}
\end{example}

\begin{exercise}
Identify all the quantifiers from Example \ref{ex:def limit} and any logical connectives.  Are there any implicit bound variables?
\end{exercise}

In order to study the abstract nature of complicated mathematical statements, it is useful to adopt some notation.

\begin{definition}
We use the symbol $\forall$ to denote the universal quantifier ``for all" and the symbol $\exists$ to denote the existential quantifier ``there exists\ldots such that".\footnote{The \TeX\ symbol commands are \texttt{\textbackslash forall} and \texttt{\textbackslash exists}, respectively.}
\end{definition}

Using our abbreviations for the logical connectives and quantifiers, we can symbolically represent mathematical propositions.

\begin{example}
For each of the following, suppose our universe of discourse is the set of real numbers.
\begin{enumerate}

\item Consider the following (true) proposition:

\begin{quote}
There exists $x$ such that $x^2-1=0$.
\end{quote}

This proposition can be denoted symbolically as $(\exists x)(x^2-1=0)$.

\item Consider the following (false) proposition:

\begin{quote}
For all $x\in \mathbb{N}$, there exists $y\in\mathbb{N}$ such that $y<x$.
\end{quote}

This one can be represented symbolically as $(\forall x)(x\in\mathbb{N}\implies (\exists y)(y\in\mathbb{N}\implies y<x))$ or more simply as $(\forall x\in\mathbb{N})(\exists y\in\mathbb{N})(y<x)$.

\item Consider the following (true) proposition:

\begin{quote}
Every positive real number has a multiplicative inverse.
\end{quote}

There are several ways of representing this statement symbolically.  However, if you unpack what a multiplicative inverse is, you'll get something like $(\forall x)(x>0 \implies (\exists y)(xy=1))$.  Alternatively, you can shorten the statement to $(\forall x>0)(\exists y)(xy=1)$.

\end{enumerate}
\end{example}

\begin{exercise} Convert the following statements into statements using only logical symbols.  Assume that the universe of discourse is the set of real numbers.
\begin{enumerate}
\item There exists a number $x$ such that $x^2+1$ is greater than zero.
\item There exists a natural number $n$ such that $n^2=36$. 
\item For every real number $x$, $x^2$ is greater than or equal to zero.
\end{enumerate}
\end{exercise}

\begin{exercise}
Express the definition of the limit in Example \ref{ex:def limit} using only logic symbols.
\end{exercise}

\begin{remark}\label{rem:implicit universal}
If $A(x)$ and $B(x)$ are predicates, then it is standard practice for the statement $A(x)\implies B(x)$ to mean $(\forall x)(A(x)\implies B(x))$ (where the universe of discourse for $x$ needs to be made clear).  In this case, we say that the universal quantifier is implicit.
\end{remark}

\begin{exercise}
Find at least two examples earlier in the notes that exhibit the claim made in Remark \ref{rem:implicit universal}.  Attempt to write the statements symbolically using explicit quantifiers.
\end{exercise}

\begin{exercise}
Convert the following proposition into a statement using only logical symbols.  The universe of discourse is the set of real numbers.  (Watch out for implicit quantifiers.)
\begin{quote}
If $\epsilon >0$, then there exists $N\in\mathbb{N}$ such that $1/N<\epsilon$.
\end{quote}
Is this statement true?
\end{exercise}

\begin{exercise}
In the last exercise, you should end up with more than one quantifier.  Reverse the order of the quantifiers to get a new statement.  Does the meaning of the statement change?  If so, how does it change?  Is the new statement true?
\end{exercise}

\begin{remark}
The symbolic expression $(\forall x)(\forall y)$ can be replaced with the simpler expression $(\forall x,y)$ as long as $x$ and $y$ are coming from the same set.
\end{remark}

\begin{exercise}
For each of the following statements, (i) unpack the statement into words, and (ii) determine whether the statement is true or false.

\begin{enumerate}
\item $(\forall n \in \mathbb{N})(n^2 \geq 5)$
\item $(\exists n \in \mathbb{N})(n^2-1=0)$
\item $(\exists N \in \mathbb{N})(\forall  n > N)(\frac{1}{n} < 0.01)$
\item $(\forall m, n \in \mathbb{Z})(2|m \wedge 2|n \implies 2|(m+n))$
\item $(\forall x \in \mathbb{N})(\exists y \in \mathbb{N})(x-2y=0)$
\item $(\exists x \in \mathbb{N})(\forall y \in \mathbb{N})(y \leq x)$
\end{enumerate}
\end{exercise}

To whet your appetite for the next section, tackle the following questions.

\begin{question}
If a statement is false, then its negation is true.  How would you go about negating a statement involving quantifiers?  In particular, if $P(x)$ is a predicate, what are the negations of $(\forall x)(P(x))$ and $(\exists x)(P(x))$, respectively?
\end{question}

\end{section}


%%% Local Variables: 
%%% mode: latex
%%% TeX-master: "IntroToProof"
%%% End: 

%\input{1.6-quantification3}
\end{chapter}

\begin{comment}
\begin{chapter}{Set Theory and Topology}
%\input{IntroToProofNotes2.1}
%\documentclass[11pt]{article}

\usepackage{amsfonts}
\usepackage{amsmath}
\usepackage{amssymb}
\usepackage{stmaryrd}
\usepackage{amsthm}
\usepackage{fancyhdr}
\usepackage[margin=1in]{geometry}
\usepackage[hang,flushmargin,symbol*]{footmisc}
\usepackage{color}
\definecolor{darkblue}{rgb}{0, 0, .6}
\definecolor{grey}{rgb}{.7, .7, .7}
\usepackage[breaklinks]{hyperref}
\hypersetup{
	colorlinks=true,
	linkcolor=darkblue,
	anchorcolor=darkblue,
	citecolor=darkblue,
	pagecolor=darkblue,
	urlcolor=darkblue,
	pdftitle={},
	pdfauthor={}
}

\pagestyle{fancy}

\lhead{\scriptsize Notes for an Introduction to Proof Course (Version Spring 2013)} 
\rhead{\scriptsize Instructor: \href{http://danaernst.com}{D.C. Ernst}}
\lfoot{\scriptsize This work is an adaptation of notes written by Stan Yoshinobu of Cal Poly and Matthew Jones of California State University, Dominguez Hills.} 
\cfoot{}
\renewcommand{\headrulewidth}{0.4pt} 
\renewcommand{\footrulewidth}{0.4pt} 

\theoremstyle{definition}
\newtheorem{theorem}{Theorem}[section]
\newtheorem{acknowledgement}[theorem]{Acknowledgement}
\newtheorem{algorithm}[theorem]{Algorithm}
\newtheorem{axiom}[theorem]{Axiom}
\newtheorem{case}[theorem]{Case}
\newtheorem{claim}[theorem]{Claim}
\newtheorem{conclusion}[theorem]{Conclusion}
\newtheorem{condition}[theorem]{Condition}
\newtheorem{conjecture}[theorem]{Conjecture}
\newtheorem{corollary}[theorem]{Corollary}
\newtheorem{criterion}[theorem]{Criterion}
\newtheorem{definition}[theorem]{Definition}
\newtheorem{example}[theorem]{Example}
\newtheorem{exercise}[theorem]{Exercise}
\newtheorem{journal}[theorem]{Journal}
\newtheorem{lemma}[theorem]{Lemma}
\newtheorem{notation}[theorem]{Notation}
\newtheorem{problem}[theorem]{Problem}
\newtheorem{proposition}[theorem]{Proposition}
\newtheorem{remark}[theorem]{Remark}
\newtheorem{solution}[theorem]{Solution}
\newtheorem{summary}[theorem]{Summary}
\newtheorem{question}[theorem]{Question}

\begin{document}

\addtocounter{section}{1}

\begin{section}{Set Theory and Topology (Continued)}

\addtocounter{subsection}{1}
\addtocounter{theorem}{27}

\begin{subsection}{Power Sets}

We've already seen that using union, intersection, set difference, and complement that we can create new sets (in the same universe) from existing sets.  In this section, we will describe another way to generate new sets; however, the new sets will not ``live" in the same universe this time.

\begin{definition}
If $S$ is a set, then the \textbf{power set} of $S$ is the set of subsets of $S$.  The power set of $S$ is denoted $\mathcal{P}(S)$.
\end{definition}

\begin{remark}
It follows immediately from the definition that $A\subseteq S$ iff $A\in\mathcal{P}(S)$.\footnote{Recall that ``iff" is an abbreviation for `if and only if", which is a statement of the form $A\iff B$ for propositions $A$ and $B$.  Recall that this is short for both $A\implies B$ \emph{and} $B\implies A$.}  It is important to pay close attention to whether ``$\subseteq$" or ``$\in$" is the proper symbol to use.
\end{remark}

\begin{example}
If $S=\{a,b\}$, then $\mathcal{P}=\{\emptyset, \{a\}, \{b\}, S\}$.
\end{example}

\begin{question}
Implicit in the definition of power set is that $S$ is a subset of some fixed universe $U$.  What universe does it make sense for $\mathcal{P}(S)$ to live in?
\end{question}

\begin{exercise}
For each of the following sets, find the power set.
\begin{enumerate}
\item $W=\{\circ, \triangle, \square\}$
\item $O=\{a,\{a\}\}$
\item $R=\emptyset$
\item $D=\{\emptyset\}$
\end{enumerate}
\end{exercise}

\begin{conjecture}
How many subsets do you think that a set with $n$ elements has?  What if $n=0$?  You do not need to prove your conjecture at this time.  We will prove this later using mathematical induction.
\end{conjecture}

\begin{exercise}
Do your best to describe $\mathcal{P}(\mathbb{N})$.  You cannot write down all of $\mathcal{P}(\mathbb{N})$.  Why not?
\end{exercise}

\begin{remark}
It is important to realize that the concepts of \emph{element} and \emph{subset} need to be carefully delineated.  For example, consider the set $A=\{x,y\}$.  The object $x$ is an element of $A$, but the object $\{x\}$ is both a subset of $A$ and an element of $\mathcal{P}(A)$.  This can get confusing rather quickly.  Consider the set $O$ from the previous example.  The set $\{a\}$ happens to be an element of $O$, a subset of $O$, and an element of  $\mathcal{P}(O)$.
\end{remark}

\begin{theorem}[*]
Let $S$ and $T$ be sets.  Then $S\subseteq T$ iff $\mathcal{P}(S)\subseteq \mathcal{P}(T)$.\footnote{To prove this theorem, you have to write two distinct subproofs: $A\implies B$ and $B\implies A$.}
\end{theorem}

\begin{theorem}[*]
Let $S$ and $T$ be sets.  Then $\mathcal{P}(S)\cap\mathcal{P}(T)=\mathcal{P}(S\cap T)$.
\end{theorem}

\begin{theorem}[*]
Let $S$ and $T$ be sets.  Then $\mathcal{P}(S)\cup\mathcal{P}(T)\subseteq \mathcal{P}(S\cup T)$.
\end{theorem}


\begin{exercise}
Let $S$ and $T$ be sets.
\begin{enumerate}
\item Provide a counterexample to show that it is not necessarily true that $\mathcal{P}(S)\cup\mathcal{P}(T)= \mathcal{P}(S\cup T)$.
\item Is it ever true that $\mathcal{P}(S)\cup\mathcal{P}(T)= \mathcal{P}(S\cup T)$ or are $\mathcal{P}(S)\cup\mathcal{P}(T)$ and $\mathcal{P}(S\cup T)$ always different sets?
\end{enumerate}
\end{exercise}

We now turn out attention to the issue of whether there is one mother of all universal sets.  Before reading any further, consider this for a moment.  That is, is there one largest set that all other sets are a subset of?  Or, in other words, is there a set of all sets?  To help wrap our heads around this issue, consider the following riddle, known as the \textbf{Barber of Seville Paradox}.

\begin{quote}
In Seville, there is a barber who shaves all those men, and only those men, who do not shave themselves.  Who shaves the barber?
\end{quote}

\begin{problem}\label{barber}
Discuss the Barber of Seville Paradox.  Does the barber shave himself or not?
\end{problem}

Problem~\ref{barber} is an example of a \textbf{paradox}.  I haven't defined paradox.  What do you think it means?  Now, suppose that there is a set of all sets and call it $\mathcal{U}$.  Then we can write $\mathcal{U}=\{A:A\mbox{ is a set}\}$.

\begin{problem}
Given our definition of $\mathcal{U}$, explain why it is an element of itself.
\end{problem}

If we continue with this line of reasoning, it must be the case that some sets are elements of themselves and some are not.  Let $X$ be the set of all sets that are elements of themselves and let $Y$ be the set of all sets that are not elements of themselves.

\begin{question}
Does $Y$ belong to $X$ or $Y$?  Explain why this is a paradox.
\end{question}

The above paradox is one way of phrasing a paradox referred to as \textbf{Russell's paradox}.  Okay, how did we get into this mess in the first place?!  By assuming the existence of a set of all sets, we can produce all sorts of paradoxes.  The only way to avoid the paradoxes is to conclude that there is no set of all sets.  Here is some more evidence that we shouldn't assume the existence of a set of all sets.

\begin{question}
If $\mathcal{U}$ is the set of all sets, then what is the relationship between $\mathcal{U}$ and $\mathcal{P}(\mathcal{U})$?  What about $\mathcal{P}(\mathcal{P}(\mathcal{U})$?
\end{question}

The upshot is that the collection of all sets is \emph{not} a set!  Here are some additional paradoxes.

\begin{problem}
Pick any two of the paradoxes below and explain why it is a paradox.
\end{problem}

\bigskip

%The following paradoxes are from Dave Richeson.
\noindent \textbf{Librarian's Paradox.} A librarian is given the unenviable task of creating two new books for the library. Book A contains the names of all books in the library that reference themselves and Book B contains the names of all books in the library that do not reference themselves. But the librarian just created two new books for the library, so their titles must be in either Book A or Book B. Clearly Book A can be listed in Book B, but where should the librarian list Book B?\\

\noindent \textbf{Liar's Paradox.} Consider the statement: this sentence is false. Is it true or false?\\

\noindent \textbf{Berry Paradox.} Consider the claim: every natural number can be unambiguously described in fourteen words or less. It seems clear that this statement is false, but if that is so, then there is some smallest natural number which cannot be unambiguously described in fourteen words or less. Let's call it $n$. But now $n$ is ``the smallest natural number that cannot be unambiguously described in fourteen words or less.'' This is a complete and unambiguous description of $n$ in fourteen words, contradicting the fact that $n$ was supposed not to have such a description. Therefore, all natural numbers can be unambiguously described in fourteen words or less!\\

\noindent \textbf{The Naming Numbers Paradox.} Consider the claim: every natural number can be unambiguously described using no more than 50 characters (where a character is a--z, 0--9, and a ``space''). For example, we can describe 9 as ``9'' or ``nine'' or ``the square of the second prime number.'' There are only 37 characters, so we can describe at most $37^{50}$ numbers, which is very large, but not infinite. So the statement is false. However, here is a ``proof'' that it is true. Let $S$ be the set of natural numbers that can be unambiguously described using no more than 50 characters. For the sake of contradiction, suppose it is not all of $\mathbb{N}$. Then there is a smallest number $t\in\mathbb{N}-S$. We can describe $t$ as: the smallest natural number not in $S$.  Thus $t$ can be described using no more than 50 characters. So $t\in S$, a contradiction.\\

\noindent \textbf{Euathlus and Protagoras.} Euathlus wanted to become a lawyer but could not pay Protagoras. Protagoras agreed to teach him under the condition that if Euathlus won his first case, he would pay Protagoras, otherwise not. Euathlus finished his course of study and did nothing. Protagoras sued for his fee. He argued:\\

\noindent If Euathlus loses this case, then he must pay (by the judgment of the court).\\
If Euathlus wins this case, then he must pay (by the terms of the contract).\\
He must either win or lose this case.\\
Therefore Euathlus must pay me.\\

\noindent But Euathlus had learned well the art of rhetoric. He responded:\\

\noindent If I win this case, I do not have to pay (by the judgment of the court).\\
If I lose this case, I do not have to pay (by the contract).\\
I must either win or lose the case.\\
Therefore, I do not have to pay Protagoras.

\end{subsection}

\end{section}

\end{document}
%\input{IntroToProofNotes2.3}
%\documentclass[11pt]{article}

\usepackage{amsfonts}
\usepackage{amsmath}
\usepackage{amssymb}
\usepackage{stmaryrd}
\usepackage{amsthm}
\usepackage{fancyhdr}
\usepackage[margin=1in]{geometry}
\usepackage[hang,flushmargin,symbol*]{footmisc}
\usepackage{color}
\definecolor{darkblue}{rgb}{0, 0, .6}
\definecolor{grey}{rgb}{.7, .7, .7}
\usepackage[breaklinks]{hyperref}
\hypersetup{
	colorlinks=true,
	linkcolor=darkblue,
	anchorcolor=darkblue,
	citecolor=darkblue,
	pagecolor=darkblue,
	urlcolor=darkblue,
	pdftitle={},
	pdfauthor={}
}

\pagestyle{fancy}

\lhead{\scriptsize Course Notes for Logic, Proof, \& Axiomatic Systems (Spring 2011)} 
\rhead{\scriptsize Instructor: \href{http://oz.plymouth.edu/~dcernst}{D.C. Ernst}} 
\lfoot{\scriptsize This work is an adaptation of notes written by Stan Yoshinobu of Cal Poly and Matthew Jones of California State University, Dominguez Hills.} 
\cfoot{} 
\renewcommand{\headrulewidth}{0.4pt} 
\renewcommand{\footrulewidth}{0.4pt} 

\theoremstyle{definition}
\newtheorem{theorem}{Theorem}[section]
\newtheorem{acknowledgement}[theorem]{Acknowledgement}
\newtheorem{algorithm}[theorem]{Algorithm}
\newtheorem{axiom}[theorem]{Axiom}
\newtheorem{case}[theorem]{Case}
\newtheorem{claim}[theorem]{Claim}
\newtheorem{conclusion}[theorem]{Conclusion}
\newtheorem{condition}[theorem]{Condition}
\newtheorem{conjecture}[theorem]{Conjecture}
\newtheorem{corollary}[theorem]{Corollary}
\newtheorem{criterion}[theorem]{Criterion}
\newtheorem{definition}[theorem]{Definition}
\newtheorem{example}[theorem]{Example}
\newtheorem{exercise}[theorem]{Exercise}
\newtheorem{journal}[theorem]{Journal}
\newtheorem{lemma}[theorem]{Lemma}
\newtheorem{notation}[theorem]{Notation}
\newtheorem{problem}[theorem]{Problem}
\newtheorem{proposition}[theorem]{Proposition}
\newtheorem{remark}[theorem]{Remark}
\newtheorem{solution}[theorem]{Solution}
\newtheorem{summary}[theorem]{Summary}
\newtheorem{question}[theorem]{Question}

\begin{document}

\addtocounter{section}{1}

\begin{section}{Set Theory and Topology}

\addtocounter{subsection}{3}
\addtocounter{theorem}{58}

\begin{subsection}{Basic Topology of $\mathbb{R}$}

\begin{remark}
For this entire section, our universe of discourse is the set of real numbers.  You may assume all the usual basic algebraic properties of the real numbers (addition, subtraction, multiplication, division, commutative property, distribution, etc.).
\end{remark}

Recall that an \textbf{axiom} is a statement that we \emph{assume} to be true.  Here are some useful axioms of the real numbers.

\begin{axiom} If $p$ and $q$ are two different real numbers in $\mathbb{R}$, then there is a number between them. %(For example $(p+q)/2$.) 
\end{axiom}

\begin{exercise}
Given real numbers $p$ and $q$ with $p<q$, construct a real number $x$ such that $p<x<q$.  (We know such a point must exist by the previous example, but this exercise is asking you to produce an actual candidate.)
\end{exercise}

\begin{axiom} (Linear ordering) If $a$, $b$, and $c$ are real numbers, then:
\begin{enumerate}
\item If $a < b$ and $b<c$, then $a<c$;
\item Exactly one of the following is true: (i) $a < b$, (ii) $a=b$, or (iii) $a>b$.
\end{enumerate}
\end{axiom}

\begin{axiom} If $p$ is a real number, then there exists $q,r\in\mathbb{R}$ such that $q<p<r$.
\end{axiom}

\begin{axiom} (Archimedean Property) If $x$ is a real number, then either (i) $x$ is an integer or (ii) there exists an integer $n$, such that $n<x<n+1$. 
\end{axiom}

\begin{definition}
Suppose $a,b\in\mathbb{R}$ such that $a<b$.  The intervals $(a,b), (-\infty,b), (a,\infty)$ are called \textbf{open intervals} while the interval $[a,b]$ is called a \textbf{closed interval}.
\end{definition}

\begin{remark} 
In this class, we will always assume that any time we write $(a,b), [a,b], (a,b]$, or $[a,b)$ that $a<b$. 
\end{remark}

\begin{exercise} Give an example of each of the following:
\begin{enumerate}
\item an open interval
\item a closed interval
\item an interval that is neither open or closed
\item an infinite set that is not an interval
\end{enumerate}
\end{exercise}

\begin{definition} A set $U$ is called an \textbf{open set} iff for every $t \in U$, there exists an open interval containing $t$ such that the open interval is a subset of $U$.  We define the empty set to be open.\end{definition}

%where should this go?
%\begin{proposition} 
%Given a number $\varepsilon >0$, there exists a natural number $N$ such that $\frac{1}{N}<\varepsilon$.
%\end{proposition}

\begin{problem} 
Prove that the set $I=(1,2)$ is an open set.
\end{problem}

\begin{theorem}[*]
Every open interval is an open set. 
\end{theorem}

\begin{theorem}
The real numbers form an open set.
\end{theorem}

\begin{theorem}[*]
Every closed interval is not an open set.
\end{theorem}

\begin{theorem}
Let $x\in\mathbb{R}$.  Then the set $\{x\}$ is not open.
\end{theorem}

\begin{exercise} 
Determine whether $\{4,17,42\}$ is an open set, and briefly justify your assertion. 
\end{exercise}

\begin{theorem}[*]\label{finite union of open sets}
Let $A$ and $B$ be open sets.  Then $A\cup B$ is an open set. 
\end{theorem}

\begin{theorem}[*]\label{finite intersection of open sets}
Let $A$ and $B$ be open sets.  Then $A\cap B$ is an open set.
\end{theorem}

\begin{theorem}[*]\label{union of open sets}
Let $\{U_{\alpha}\}_{\alpha\in\Delta}$ be a collection of open sets.  Then
\[
\bigcup_{\alpha\in\Delta} U_{\alpha}
\]
is an open set.
\end{theorem}

\begin{exercise}\label{intersection of open sets}\
\begin{enumerate}
\item Find a collection of open sets $\{U_{\alpha}\}_{\alpha\in\Delta}$ such that
\[
\bigcap_{\alpha\in\Delta} U_{\alpha}
\]
is not an open set.
\item Find a collection of open sets $\{B_{\alpha}\}_{\alpha\in\Delta}$ such that
\[
\bigcap_{\alpha\in\Delta} B_{\alpha}
\]
is an open set.
\end{enumerate}
\end{exercise}

\begin{remark}
Taken together, Theorems \ref{finite union of open sets}--\ref{union of open sets} and Exercise \ref{intersection of open sets} tell us that the union of open sets is open, but that the intersection of open sets may or may not be open.  However, if we are taking the intersection of finitely many open sets, then the intersection will be open.
\end{remark}

\begin{exercise}
Provide an example of an open set that is not an open interval.
\end{exercise}

\begin{exercise}
Determine whether each of the following sets are open or not open.
\begin{enumerate}
\item $\displaystyle W=\bigcup_{n=2}^{\infty} \left(n - \frac{1}{2},n\right)$
\item $\displaystyle X=\bigcap_{n=1}^{\infty} \left(-\frac{1}{n}, \frac{1}{n}\right)$
\end{enumerate}
\end{exercise}

\begin{definition}
A point $p$ is a \textbf{limit point of the set $S$} iff for every open interval $I$ containing $p$, there exists a point $q \in I$ such that $q \in S$ with $q\neq p$.
\end{definition}

\begin{problem}
Consider the open interval $S=(1,2)$. Prove each of the following.
\begin{enumerate}
\item The point $p=2$ is a limit point of $S$.
\item If $p\in S$, then $p$ is a limit point of $S$.
\item If $p<1$ or $p>2$, then $p$ is not a limit point of $S$.
\end{enumerate}
\end{problem}

\begin{theorem}[*]
A point $p$ is a limit point of $(a,b)$ iff $p\in [a,b]$.
\end{theorem}

\begin{problem}
Prove that the point $p=0$ is a limit point of $S=\{\frac{1}{n}: n \in \mathbb{N}\}$.  Are there any other limit points?
\end{problem}

\begin{exercise}
Provide an example of a set $S$ such that 1 is a limit point of $S$, $1\neq S$, and $S$ contains no intervals.
\end{exercise}

\begin{exercise}
Provide an example of a set $T$ with exactly two limit points.
\end{exercise}

\begin{theorem}
If $p\in\mathbb{R}$, then $p$ is a limit point of $\mathbb{Q}$.
\end{theorem}

\begin{definition}
A set is called \textbf{closed} iff it contains all of its limit points.
\end{definition}

\begin{exercise}
Provide an example of each of the following.  You do not need to prove that your answers are correct.
\begin{enumerate}
\item A closed set.
\item A set that is not closed.
\item A set that is open and closed.
\item A set that neither open or closed.
\end{enumerate}

\end{exercise}

\begin{theorem}[*]
The set $[a,b]$ is closed.
\end{theorem}

\begin{theorem}
The set $U$ is open iff $U^C$ is closed.
\end{theorem}

\begin{theorem}[*]
Every finite set is closed.
\end{theorem}

\begin{theorem}
The real numbers are both open and closed.
\end{theorem}

\begin{theorem}
The rational numbers are neither open or closed.  (You may use the fact that between any two rational numbers, there exists an irrational number.)
\end{theorem}

\begin{theorem}
The empty set is both open and closed.
\end{theorem}

\begin{theorem}[*]
Let $\{A_{\alpha}\}_{\alpha\in\Delta}$ be a collection of closed sets.  Then
\[
\bigcap_{\alpha\in \Delta} A_{\alpha}
\]
is a closed set.
\end{theorem}

\begin{exercise}
Provide an example of a collection of closed sets $\{A_{\alpha}\}_{\alpha\in\Delta}$ such that 
\[
\bigcup_{\alpha\in \Delta} A_{\alpha}
\]
is a \emph{not} closed set.
\end{exercise}

\begin{problem}
Let $A$ and $B$ be closed sets.  Determine whether $A\cup B$ is necessarily closed and prove that your answer is correct.
\end{problem}

\end{subsection}

\end{section}

\end{document}

\end{chapter}

\begin{chapter}{Induction}
%\documentclass[11pt]{article}

\usepackage{amsfonts}
\usepackage{amsmath, amsthm}
\usepackage{wasysym}
\usepackage{amssymb}
\usepackage{stmaryrd}
\usepackage{amsthm}
\usepackage{fancyhdr}
\usepackage[margin=1in]{geometry}
\usepackage[hang,flushmargin,symbol*]{footmisc}
\usepackage{color}
\definecolor{darkblue}{rgb}{0, 0, .6}
\definecolor{grey}{rgb}{.7, .7, .7}
\usepackage[breaklinks]{hyperref}
\hypersetup{
	colorlinks=true,
	linkcolor=darkblue,
	anchorcolor=darkblue,
	citecolor=darkblue,
	pagecolor=darkblue,
	urlcolor=darkblue,
	pdftitle={},
	pdfauthor={}
}

\newcommand{\dom}{\operatorname{Dom}}
\newcommand{\codom}{\operatorname{Codom}}
\newcommand{\range}{\operatorname{Rng}}

\pagestyle{fancy}

\lhead{\scriptsize Notes for an Introduction to Proof Course (Version Spring 2013)}
\rhead{\scriptsize Instructor: \href{http://danaernst.com}{D.C. Ernst}}
\lfoot{\scriptsize This work is an adaptation of notes written by Stan Yoshinobu of Cal Poly and Matthew Jones of California State University, Dominguez Hills.} 
\cfoot{}
\renewcommand{\headrulewidth}{0.4pt} 
\renewcommand{\footrulewidth}{0.4pt}


\theoremstyle{definition}
\newtheorem{theorem}{Theorem}[section]
\newtheorem{acknowledgement}[theorem]{Acknowledgement}
\newtheorem{algorithm}[theorem]{Algorithm}
\newtheorem{axiom}[theorem]{Axiom}
\newtheorem{case}[theorem]{Case}
\newtheorem{claim}[theorem]{Claim}
\newtheorem{conclusion}[theorem]{Conclusion}
\newtheorem{condition}[theorem]{Condition}
\newtheorem{conjecture}[theorem]{Conjecture}
\newtheorem{corollary}[theorem]{Corollary}
\newtheorem{criterion}[theorem]{Criterion}
\newtheorem{definition}[theorem]{Definition}
\newtheorem{example}[theorem]{Example}
\newtheorem{exercise}[theorem]{Exercise}
\newtheorem{journal}[theorem]{Journal}
\newtheorem{lemma}[theorem]{Lemma}
\newtheorem{notation}[theorem]{Notation}
\newtheorem{problem}[theorem]{Problem}
\newtheorem{proposition}[theorem]{Proposition}
\newtheorem{remark}[theorem]{Remark}
\newtheorem{solution}[theorem]{Solution}
\newtheorem{summary}[theorem]{Summary}
\newtheorem{question}[theorem]{Question}
\newtheorem{skeleton}[theorem]{Skeleton Proof}

\newsavebox{\savepar}
\newenvironment{textbox}{\noindent\begin{lrbox}{\savepar}\begin{minipage}[c]{.98\textwidth}}{\end{minipage}\end{lrbox}\fcolorbox{black}{white}{\usebox{\savepar}}}

\begin{document}

\addtocounter{section}{2}

\begin{section}{Induction}

\begin{subsection}{Introduction to Induction}

In this section, we will explore a technique for proving statements of the form $(\forall n \in \mathbb{N})P(n)$, where $P(n)$ is some predicate.  Notice that this is a statement about natural numbers and not some other set.  Consider the claims:
\begin{enumerate}
\item For all $n\in\mathbb{N}$, $\displaystyle 1+2+3+\cdots +n=\frac{n(n+1)}{2}$.
\item For all $n\in\mathbb{N}$, $n^{2}+n+41$ is prime.
\end{enumerate}
Let's take a look at potential proofs.

\bigskip

\noindent \emph{``Proof'' of Claim (1).} If $n=1$, then $1=\frac{1(1+1)}{2}$.  If $n=2$, then $1+2=3=\frac{2(2+1)}{2}$.  If $n=3$, then $1+2+3=6=\frac{3(3+1)}{2}$, and so on. \hfill \qed

\bigskip

\noindent \emph{``Proof'' of Claim (2).} If $n=1$, then $n^{2}+n+41=43$, which is prime.  If $n=2$, then $n^{2}+n+41=47$, which is prime.  If $n=3$, then $n^{2}+n+41=53$, which is prime, and so on. \hfill \qed

\bigskip

Are these actual proofs?  The answer is NO!  In fact, the second claim isn't even true.  If $n=41$, then $n^{2}+n+41=41^{2}+41+41=41(41+1+1)$, which is not prime since it has 41 as a factor.  It turns out that the first claim is true, but what we wrote cannot be a proof since the same type of reasoning when applied to the second claim seems to prove something that isn't actually true.  We need a rigorous way of capturing ``and so on'' and a way to verify whether it really is ``and so on.''

\begin{axiom}[Axiom of Induction]
Let $S\subseteq \mathbb{N}$ such that both
\begin{enumerate}
\item $1\in S$, and
\item if $k\in S$, then $k+1\in S$.
\end{enumerate}
Then $S=\mathbb{N}$.
\end{axiom}

\begin{remark}
Recall that an axiom is a basic mathematical assumption.  That is, we are assuming that the Axiom of Induction is true, which I'm hoping that you can agree is a pretty reasonable assumption.  I like to think of the first hypothesis of the Axiom of Induction as saying that we have a first rung of a ladder.  The second hypothesis says that if we have some random rung, we can always get to the next rung.  Taken together, this says that we can get from the first rung to the second, from the second to the third, and so on.  Again, we are assuming that the ``and so on'' works as expected here.
\end{remark}

\begin{theorem}[Principle of Mathematical Induction, *]
Let $P(1), P(2), P(3), \ldots$ be a sequence of statements, one for each natural number.\footnote{In this case, you should think of $P(n)$ as a predicate, where $P(1)$ is the statement that corresponds to substituting in the value 1 for $n$.} Assume
\begin{enumerate}
\item $P(1)$ is true, and
\item If $P(k)$ is true, then $P(k+1)$ is true.
\end{enumerate}
Then $P(n)$ is true for all $n\in\mathbb{N}$.\footnote{\emph{Hint:} Let $S=\{k\in \mathbb{N}: P_k \text{ is true}\}$ and use the Axiom of Induction.  The set $S$ is sometimes called the \emph{truth set}.  Your job is to show that the truth set is all of $\mathbb{N}$.}
\end{theorem}

\begin{remark}
The Principal of Mathematical Induction (PMI) provides us with a process for proving statements of the form: ``For all $n\in\mathbb{N}$, $P(n)$,'' where $P(n)$ is some predicate involving $n$.  Hypothesis (1) above is called the \textbf{base step} while (2) is called the \textbf{inductive step}.
\end{remark} 

\begin{skeleton}[Proof by induction for $(\forall n\in\mathbb{N})P(n)$]
Here is what the general structure for a proof by induction looks like.  Remarks are in parentheses.

\bigskip

\begin{textbox}
\begin{proof}
We proceed by induction.
\begin{enumerate}
\item[(i)] Base step: (Verify that $P(1)$ is true.  This often, but not always, amounts to plugging $n=1$ into two sides of some claimed equation and verifying that both sides are actually equal.  Don't assume that they are equal!)

\item[(ii)] Inductive step:  (Your goal is to prove that ``For all $k\in\mathbb{N}$, if $P(k)$ is true, then $P(k+1)$ is true.'')  Let $k\in\mathbb{N}$ and assume that $P(k)$ is true.  (Now, do some stuff to show that $P(k+1)$ is true.)  Therefore, $P(k+1)$ is true.
\end{enumerate}
Thus, by the PMI, $P(n)$ is true for all $n\in\mathbb{N}$.
\end{proof}
\end{textbox}

\end{skeleton}

\begin{theorem}[*]
For all $n\in\mathbb{N}$, $\displaystyle \sum_{i=1}^{n}i=\frac{n(n+1)}{2}$.\footnote{Recall that $\displaystyle \sum_{i=1}^{n}i=1+2+3+\cdots +n$, by definition.  Also, this theorem should look familar from calculus.}
\end{theorem}

\begin{theorem}[*]
For all $n\in\mathbb{N}$, 3 divides $4^{n}-1$.
\end{theorem}

\begin{theorem}[*]
For all $n\in\mathbb{N}$, 6 divides $n^{3}-n$.
\end{theorem}

\begin{theorem}[*]
Let $p_{1}, p_{2}, \ldots, p_{n}$ be $n$ distinct points arranged on a circle.  Then the number of line segments joining all pairs of points is $\frac{n^{2}-n}{2}$.
\end{theorem}

\begin{theorem}[*]
Let $A$ be a set with $n$ elements.  Then $\mathcal{P}(A)$ is a set with $2^{n}$ elements.\footnote{We encountered this theorem back in Section 2.2 (see Conjecture 2.33), but we didn't prove it.  Proving this theorem is rather tricky.  If you use induction (which I suggest), at some point, you will need to argue that if you add one more element to a finite set, then you end up with twice as many subsets.}
\end{theorem}

\end{subsection}

\end{section}

\end{document}
\end{chapter}

\begin{chapter}{Relations and Functions}
%\documentclass[11pt]{article}

\usepackage{amsfonts}
\usepackage{amsmath, amsthm}
\usepackage{wasysym}
\usepackage{graphicx}
\usepackage{amssymb}
\usepackage{stmaryrd}
\usepackage{amsthm}
\usepackage{fancyhdr}
\usepackage[margin=1in]{geometry}
\usepackage[hang,flushmargin,symbol*]{footmisc}
\usepackage{color}
\definecolor{darkblue}{rgb}{0, 0, .6}
\definecolor{grey}{rgb}{.7, .7, .7}
\usepackage[breaklinks]{hyperref}
\hypersetup{
	colorlinks=true,
	linkcolor=darkblue,
	anchorcolor=darkblue,
	citecolor=darkblue,
	pagecolor=darkblue,
	urlcolor=darkblue,
	pdftitle={},
	pdfauthor={}
}

\newcommand{\dom}{\operatorname{Dom}}
\newcommand{\codom}{\operatorname{Codom}}
\newcommand{\range}{\operatorname{Rng}}

\newcommand{\ndv}{\hspace{-4pt}\not|\hspace{2pt}}

\pagestyle{fancy}

\lhead{\scriptsize Notes for an Introduction to Proof Course (Version Spring 2013)}
\rhead{\scriptsize Instructor: \href{http://danaernst.com}{D.C. Ernst}}
\lfoot{\scriptsize This work is an adaptation of notes written by Stan Yoshinobu of Cal Poly and Matthew Jones of California State University, Dominguez Hills.} 
\cfoot{}
\renewcommand{\headrulewidth}{0.4pt} 
\renewcommand{\footrulewidth}{0.4pt}


\theoremstyle{definition}
\newtheorem{theorem}{Theorem}[section]
\newtheorem{acknowledgement}[theorem]{Acknowledgement}
\newtheorem{algorithm}[theorem]{Algorithm}
\newtheorem{axiom}[theorem]{Axiom}
\newtheorem{case}[theorem]{Case}
\newtheorem{claim}[theorem]{Claim}
\newtheorem{conclusion}[theorem]{Conclusion}
\newtheorem{condition}[theorem]{Condition}
\newtheorem{conjecture}[theorem]{Conjecture}
\newtheorem{corollary}[theorem]{Corollary}
\newtheorem{criterion}[theorem]{Criterion}
\newtheorem{definition}[theorem]{Definition}
\newtheorem{example}[theorem]{Example}
\newtheorem{exercise}[theorem]{Exercise}
\newtheorem{journal}[theorem]{Journal}
\newtheorem{lemma}[theorem]{Lemma}
\newtheorem{notation}[theorem]{Notation}
\newtheorem{problem}[theorem]{Problem}
\newtheorem{proposition}[theorem]{Proposition}
\newtheorem{remark}[theorem]{Remark}
\newtheorem{solution}[theorem]{Solution}
\newtheorem{summary}[theorem]{Summary}
\newtheorem{question}[theorem]{Question}
\newtheorem{skeleton}[theorem]{Skeleton Proof}

\newsavebox{\savepar}
\newenvironment{textbox}{\noindent\begin{lrbox}{\savepar}\begin{minipage}[c]{.98\textwidth}}{\end{minipage}\end{lrbox}\fcolorbox{black}{white}{\usebox{\savepar}}}

\begin{document}

\addtocounter{section}{3}

\begin{section}{Two Famous Theorems}

\begin{subsection}{The infinitude of primes}

The highlight of this section\footnote{This section is derived from work of \href{http://users.dickinson.edu/~richesod/}{Dave Richeson} of Dickinson College.} is Theorem~\ref{thm:infprimes}, which states that there are infinitely primes. In case you forgot, here is the definition of a prime number.

\begin{definition}
A natural number $p$ is called \textbf{prime} iff $p$ is divisible by exactly two distinct natural numbers (namely, 1 and $p$ itself).
\end{definition}

\begin{exercise}
Is 1 a prime number?  Explain your answer.
\end{exercise}

The first known proof of Theorem~\ref{thm:infprimes} is in Eulcid's \emph{Elements} (c.\ 300 BCE). Euclid stated it as follows: 
\begin{quote}
\textbf{Proposition IX.20.} Prime numbers are more than any assigned multitude of prime numbers.
\end{quote}
There are a few interesting observations to make about Euclid's proposition and his proof. First, notice that the statement of the theorem does not contain the word ``infinity.'' The Greek's were skittish about the idea of infinity. Thus he proved that there were more primes than any given finite number. Today we'd say that they are infinite. In fact, Euclid proved that there are more than \emph{three} primes and concluded that there were more than any finite number. While you would lose points for such a proof in this class, we can forgive Euclid for this less-than-rigorous proof;  in fact, it is easy to turn his proof into the general one that you will give below. Lastly, Euclid's proof was geometric. He was viewing his numbers as line segments with integral length. The modern concept of number was not developed yet.

Prior to tackling a proof of Theorem~\ref{thm:infprimes}, we need to prove a couple lemmas.  The proof of the first lemma is provided for you. 

\begin{lemma}\label{lem:divisorsof1}
The only natural number that divides $1$ is $1$.  
\end{lemma}

\begin{proof}
Let $m$ be a natural number that divides $1$. We know that $m\geq 1$ because 1 is the smallest positive integer. Since $m$ divides $1$, there exists $k\in \mathbb{N}$ such that $1=mk$. Since $k\geq 1$, we see that $mk\geq m$.  But $1=mk$, and so $1\geq m$.  Thus, we have $1\leq m \leq 1$, which implies that $m=1$, as desired.
\end{proof}

\begin{lemma}\label{lem:plus1}
Let $p$ be a prime number  and let $n\in \mathbb{Z}$. If $p$ divides $n$, then $p$ does not divide $n+1$.\footnote{\emph{Hint:} Use a proof by contradiction and utilize the previous lemma.}
\end{lemma}

Now, we are ready to prove the following important theorem.

\begin{theorem}\label{thm:infprimes}
There are infinitely many prime numbers.\footnote{\emph{Hint:} Use a proof by contradiction.  In this case, there are finitely many primes.  Consider the product of all of them and then add 1.}
\end{theorem}

\end{subsection}

\end{section}

\end{document}
%\documentclass[11pt]{article}

\usepackage{amsfonts}
\usepackage{amsmath, amsthm}
\usepackage{wasysym}
\usepackage{graphicx}
\usepackage{amssymb}
\usepackage{stmaryrd}
\usepackage{amsthm}
\usepackage{fancyhdr}
\usepackage[margin=1in]{geometry}
\usepackage[hang,flushmargin,symbol*]{footmisc}
\usepackage{color}
\definecolor{darkblue}{rgb}{0, 0, .6}
\definecolor{grey}{rgb}{.7, .7, .7}
\usepackage[breaklinks]{hyperref}
\hypersetup{
	colorlinks=true,
	linkcolor=darkblue,
	anchorcolor=darkblue,
	citecolor=darkblue,
	pagecolor=darkblue,
	urlcolor=darkblue,
	pdftitle={},
	pdfauthor={}
}

\newcommand{\dom}{\operatorname{Dom}}
\newcommand{\codom}{\operatorname{Codom}}
\newcommand{\range}{\operatorname{Rng}}

\pagestyle{fancy}

\lhead{\scriptsize Notes for an Introduction to Proof Course (Version Spring 2013)}
\rhead{\scriptsize Instructor: \href{http://danaernst.com}{D.C. Ernst}}
\lfoot{\scriptsize This work is an adaptation of notes written by Stan Yoshinobu of Cal Poly and Matthew Jones of California State University, Dominguez Hills.} 
\cfoot{}
\renewcommand{\headrulewidth}{0.4pt} 
\renewcommand{\footrulewidth}{0.4pt}


\theoremstyle{definition}
\newtheorem{theorem}{Theorem}[section]
\newtheorem{acknowledgement}[theorem]{Acknowledgement}
\newtheorem{algorithm}[theorem]{Algorithm}
\newtheorem{axiom}[theorem]{Axiom}
\newtheorem{case}[theorem]{Case}
\newtheorem{claim}[theorem]{Claim}
\newtheorem{conclusion}[theorem]{Conclusion}
\newtheorem{condition}[theorem]{Condition}
\newtheorem{conjecture}[theorem]{Conjecture}
\newtheorem{corollary}[theorem]{Corollary}
\newtheorem{criterion}[theorem]{Criterion}
\newtheorem{definition}[theorem]{Definition}
\newtheorem{example}[theorem]{Example}
\newtheorem{exercise}[theorem]{Exercise}
\newtheorem{journal}[theorem]{Journal}
\newtheorem{lemma}[theorem]{Lemma}
\newtheorem{notation}[theorem]{Notation}
\newtheorem{problem}[theorem]{Problem}
\newtheorem{proposition}[theorem]{Proposition}
\newtheorem{remark}[theorem]{Remark}
\newtheorem{solution}[theorem]{Solution}
\newtheorem{summary}[theorem]{Summary}
\newtheorem{question}[theorem]{Question}
\newtheorem{skeleton}[theorem]{Skeleton Proof}

\newsavebox{\savepar}
\newenvironment{textbox}{\noindent\begin{lrbox}{\savepar}\begin{minipage}[c]{.98\textwidth}}{\end{minipage}\end{lrbox}\fcolorbox{black}{white}{\usebox{\savepar}}}

\begin{document}

\addtocounter{section}{3}

\begin{section}{Two Famous Theorems (continued)}

\addtocounter{subsection}{1}
\addtocounter{theorem}{5}

\begin{subsection}{The irrationality of $\sqrt{2}$}

In this section\footnote{This section is derived from work of \href{http://users.dickinson.edu/~richesod/}{Dave Richeson} of Dickinson College.} we will prove one of the oldest and most important theorems in mathematics. The Pythagoreans were an ancient secret society that followed their spiritual leader: Pythagoras of Samos (c.\ 570-495 BCE). The Pythagoreans believed that the way to spiritual fulfillment and to an understanding of the universe was through the study of mathematics. They believed that all of mathematics, music, and astronomy could be described via whole numbers and their ratios. In modern mathematical terms they believed that all numbers are rational. Attributed to Pythagoras is the saying, ``Beatitude is the knowledge of the perfection of the numbers of the soul.'' And their motto was ``All is number.''

Thus they were stunned when one of their own---Hippasus of Metapontum (c.\ 5th century BCE)---discovered that the side and the diagonal of a square are incommensurable. That is, the ratio of the length of the diagonal to the length of the side is irrational\footnote{Recall that a number is \textbf{rational} if it can be written in the form $\frac{m}{n}$, where $m,n\in\mathbb{Z}$ and $n\neq 0$.  A number is \textbf{irrational} if it is not rational.}. Indeed, if the side of the square has length $a$, then the diagonal will have length $a\sqrt{2}$; the ratio is $\sqrt{2}$ (see Figure~\ref{fig:square}).  In today's language, Hipassus discovery is given by the following theorem.

\begin{figure}[ht]
\begin{center}
\includegraphics{square}
\end{center}
\vspace{-.5cm}
\caption{The side and diagonal of a square are incommensurable.}
\label{fig:square}
\end{figure}

Before tackling a proof of Theorem~\ref{thm:sqrt2}, we need a few tools.  In particular, we will make use of the Fundamental Theorem of Arithmetic (see Corollary~\ref{cor:FTA}).  The following result makes up half of the Fundamental Theorem of Arithmetic.

\begin{theorem}[*]\label{thm:prodprimes}
Let $n$ be a natural number greater than 1.  Then $n$ can be expressed as a product of primes.  That is, we can write
\[
n=p_1 p_2 \cdots p_k,
\]
where each of $p_1, p_2, \ldots, p_k$ are prime numbers (and there may possibly be repeats in this list).\footnote{\emph{Hint:} Use a proof by contradiction.  Let $n$ be the smallest natural number for which the theorem fails.  Then $n$ cannot be prime since this would satisfy the theorem.  So, it must be the case that $n$ has a divisor other than 1 and itself.  This implies that there exists natural numbers $a$ and $b$ greater than 1 such that $n=ab$.  Since $n$ was our smallest counterexample, what can you conclude about both $a$ and $b$?  Use this information to derive a counterexample for $n$.}
\end{theorem}

The previous theorem states that we can write every natural number as a product of primes, but it does not say is that the primes and the number of times the primes appear are unique.  It turns out that this is fairly difficult to prove.  We will need the following result known as the Division Algorithm, but we won't worry about proving it.  Instead, we will take it for granted and use it in the proof of Theorem~\ref{thm:Euclid}, which we will then use to prove uniqueness

\begin{theorem}[Division Algorithm]
Suppose that $m,n\in\mathbb{N}$.  Then there exists unique $q,r\in\mathbb{N}$ such that $m=nq+r$ with $0\leq r<n$.
\end{theorem}
The numbers $q$ and $r$ from the Division Algorithm are referred to as \textbf{quotient} and \textbf{remainder}, respectively.  Now, see if you can prove the following theorem, which is known as Euclid's Lemma.

\begin{theorem}[Euclid's Lemma, *]\label{thm:Euclid}
Assume that $p$ is prime.  If $p$ divides $ab$, where $a,b\in\mathbb{N}$, then either $p$ divides $a$ or $p$ divides $b$.\footnote{\emph{Hint:} Use a proof by contradiction and apply the Division Algorithm to both $a$ and $b$.  What can you say about $ab$?}
\end{theorem}

Alright, let's tackle the uniqueness of the product of primes now.

\begin{theorem}[*]\label{thm:unique}
Let $n$ be a natural number greater than 1.  Then the expression for $n$ as the product of one or more primes is unique (up to the order in which they appear).\footnote{\emph{Hint:} Use a proof by contradiction.  Write $n$ as both $p_1 p_2 \cdots p_k$ and $q_1 q_2 \cdots q_l$, where both are products of primes.  Use Euclid's Lemma to derive a contradiction.}
\end{theorem}

The following corollary follows immediately from Theorem~\ref{thm:prodprimes} and Theorem~\ref{thm:unique}.

\begin{corollary}[Fundamental Theorem of Arithmetic]\label{cor:FTA}
Every natural number greater than 1 can be expressed uniquely (up to the order in which that appear) as the product of one or more primes.
\end{corollary}

We are finally ready to prove that $\sqrt{2}$ is irrational.

\begin{theorem}[*]
\label{thm:sqrt2}
The real number $\sqrt{2}$ is irrational\footnote{\emph{Hint:} Use a proof by contradiction.  That is, suppose that there exist $m,n\in\mathbb{Z}$ such that $n\ne 0$ and $\sqrt{2}=\frac{m}{n}$. Moreover, assume $m$ and $n$ have no common factors.  Next, solve for $n$ and square both sides.  Derive a contradiction using Theorem~\ref{thm:FTA}.}.
\end{theorem}

As one might expect, the Pythagoreans were unhappy with this discovery. Legend says that Hippasus was expelled from the Pythagoreans and was perhaps drowned at sea. Ironically, this result, which angered the Pythagoreans so much, is probably their greatest contribution to mathematics: the discovery of irrational numbers.

Now, let's tackle a few more problems dealing with irrational numbers.

\begin{problem}
Determine whether $\displaystyle \frac{1+\sqrt{2}}{3+2\sqrt{2}}$ is rational or irrational and then prove that your answer is correct.
\end{problem}

\begin{theorem}[*]\label{thm:sqrtp}
Let $p$ be a prime number.  Then $\sqrt{p}$ is irrational.
\end{theorem}

\begin{exercise}
Let $p$ be a prime number.  For which values of $n\in\mathbb{N}$ is $\sqrt[n]{p}$ irrational?  You do not need to prove your answer.
\end{exercise}

\begin{theorem}[*]\label{thm:sqrt(pq)}
Let $p$ and $q$ be distinct primes.  Then $\sqrt{pq}$ is irrational.
\end{theorem}

\begin{problem}
State a generalization of Theorem~\ref{thm:sqrt(pq)} and briefly describe how its proof would go.  Be as general as possible.
\end{problem}

\begin{remark}
It is important to point out that not every positive irrational number is equal to the square root of some natural number.  For example, $\pi$ is irrational, but is not equal to the square root of a natural number.
\end{remark}

It is worth pointing out that our approach for proving that $\sqrt{2}$ was irrational was not the most efficient.  However, our technique was easy to generalize to handle results like Theorem~\ref{thm:sqrtp}.

\end{subsection}

\end{section}

\end{document}
%\input{IntroToProofNotes4.3}
%\input{IntroToProofNotes4.4}
%\input{IntroToProofNotes4.5}
\end{chapter}
\end{comment}

\end{document}

%%% Local Variables:
%%% mode: latex
%%% TeX-master: t
%%% End:
